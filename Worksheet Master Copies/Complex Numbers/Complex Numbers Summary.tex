\documentclass{article}

\usepackage[left=2cm,right=2cm, top=2cm, bottom = 2cm]{geometry}
\usepackage{amsfonts}
\usepackage{amsmath}
\usepackage{array}
\usepackage{tikz}

\setlength{\tabcolsep}{1.8cm}
\renewcommand{\arraystretch}{2.5}

\makeatletter
\newcommand{\thickhline}{%
    \noalign {\ifnum 0=`}\fi \hrule height 2pt
    \futurelet \reserved@a \@xhline
}
\newcolumntype{!}{@{\hskip\tabcolsep\vrule width 2pt\hskip\tabcolsep}}
\makeatother

\newcommand{\cis}{\,\mathrm{cis}}

\begin{document}

\title{Complex Numbers--Summary}
\date{}

\maketitle
\thispagestyle{empty}
\pagestyle{empty}

\Large


\section{Key Points - Fill in the Blanks:}

Fill in the blanks in the key points below with a word, phrase, or mathematical expression. The unblanked versions are on the next page. Note that the size of the blank does not indicate the size of the missing word or phrase!

\begin{enumerate}
\item The \textbf{imaginary unit} $j$ (or $i$) is defined to be a solution to the equation ........ The other solution to this equation is .....
\item A \textbf{complex number} is any expression of the form $a+bj$ where $a$ and $b$ are real numbers. This is called the ......... form of the complex number; $a$ is called the ......... and $b$ is called the .........
\item If $z=a+bj$, then the complex number $\bar{z}$ is equal to .......... and is called the ............ of $z$.
\item We can represent the complex number $a+bj$ by the point $(a,b)$ in the .........
\item If $z$ is a complex number, the non-negative real number $|z|$ is called the .......... of $z$ and the angle $\arg(z)$ is called the ......... of $z$. These are the ......... of the point $(a,b)$ in the Argand diagram.
\item If $z=a+bj$, then $|z|=$........ and $\tan(\arg(z))=$........
\item The abbreviation $\cis(\theta)$ is shorthand for .............
\item We can write any complex number $z$ in the form $r\cis(\theta)$, where $r=$...... and $\theta=$....... This is called the .......... of $z$.
\item To multiply two complex numbers in polar form, we ..............................
\item \textbf{De Moivre's Theorem} states that if $z=r\cis(\theta)$, then ..........
\item When using de Moivre's Theorem with $b\neq 1$ (\textit{e.g.}, to find roots of a complex number) we must always remember ..........
\end{enumerate}

\clearpage

\section{Key Points to Remember}

The statements from the previous page, with the blanks filled in.

\begin{enumerate}
\item The \textbf{imaginary unit} $j$ (or $i$) is defined to be a solution to the equation $x^2=-1$. The other solution to this equation is $-j$.
\item A \textbf{complex number} is any expression of the form $a+bj$ where $a$ and $b$ are real numbers. This is called the \textit{cartesian} form of the complex number; $a$ is called the \textit{real part} and $b$ is called the \textit{imaginary part}.
\item If $z=a+bj$, then the complex number $\bar{z}$ is equal to $a-bj$ and is called the \textit{complex conjugate} of $z$.
\item We can represent the complex number $a+bj$ by the point $(a,b)$ in the \textit{Argand diagram} (or \textit{complex plane}).
\item If $z$ is a complex number, the non-negative real number $|z|$ is called the \textit{modulus} of $z$ and the angle $\arg(z)$ is called the \textit{argument} of $z$. These are the \textit{polar coordinates} of the point $(a,b)$ in the Argand diagram.
\item If $z=a+bj$, then $|z|=\sqrt{a^2+b^2}$ and $\tan(\arg(z))=\frac{b}{a}$.
\item The abbreviation $\cis(\theta)$ is shorthand for $\cos(\theta)+j\sin(\theta)$.
\item We can write any complex number $z$ in the form $r\cis(\theta)$, where $r=|z|$ and $\theta=\arg(z)$. This is called the \textit{polar form} of $z$.
\item To multiply two complex numbers in polar form, we \textit{multiply the moduli and add the arguments}.
\item \textbf{De Moivre's Theorem} states that if $z=r\cis(\theta)$, then $z^{a/b}=r^{a/b}\cis\left(\frac{a\theta}{b}\right)$.
\item When using de Moivre's Theorem with $b\neq 1$ (\textit{e.g.}, to find roots of a complex number) we must always remember \textit{to add multiples of $2\pi$ to the argument to get all the roots}.
\end{enumerate}


\clearpage

\section{Revision Questions}

\begin{enumerate}
\item Let $z=1-j$ and $w=4+3j$. Compute $z+w$, $zw$, $\bar{z}$, $\bar{w}$, $\frac{z}{w}$ and $z^2$.
\item Express $5-12j$ in polar form.
\item Express $14\cis\left(\frac{-5\pi}{6}\right)$ in cartesian form.
\item Let $z=\frac{1}{2}(\sqrt{3}-j)$. Compute $z^{100}$.
\item Find all cube roots of $-2+j$.
\end{enumerate}




\clearpage

\section{Solutions}

It is possible I've made a mistake or two in these, so if your answer is different from mine and after checking you can't find a mistake in your work, ask me about it!

\begin{enumerate}
\item 
	\begin{align*}
		z+w &= 5+2j\\
		zw&= 7-j\\
		\bar{z}&=1+j\\
		\bar{w}&=4-3j\\
		\frac{z}{w}&=\frac{(1-j)(4-3j)}{4^2+3^2}=\frac{1}{25} -\frac{7}{25}j\\
		z^2 &= -2j
	\end{align*}
\item $|5-12j| = \sqrt{5^2+12^2} = \sqrt{169} = 13$; $\tan(\arg(5-12j))=\frac{-12}{5}$ and we're in the bottom right quadrant, so $\arg(5-12j) = \tan^{-1}\left(\frac{-12}{5}\right)\approx -1.176$. So $5-12j=13\cis(-1.176)$.

\item The real part is $14\cos\left(\frac{-5\pi}{6}\right)$; $\frac{-5\pi}{6}$ is in the bottom left quadrant, at an angle of $\frac{\pi}{6}$ below the negative real axis, so $\cos\left(\frac{-5\pi}{6}\right)=-\cos\left(\frac{\pi}{6}\right)=-\frac{\sqrt{3}}{2}$. So $\mathrm{Re}(14\cis\left(\frac{-5\pi}{6}\right)=-7\sqrt{3}$. The imaginary part is $14\sin\left(\frac{-5\pi}{6}\right)=14\sin\left(\frac{\pi}{6}\right) = 7$ (again by considering the quadrants). So $14\cis\left(\frac{-5\pi}{6}\right) = -7\sqrt{3}+7j$.

\item We use de Moivre's Theorem. First we put $z$ into polar form; $|z|=\sqrt{\frac{3}{4}+\frac{1}{4}}=1$, $\tan(\arg(z))=\frac{-1}{\sqrt{3}}$, and $z$ is in the bottom right quadrant, so $\arg(z)=\tan^{-1}\left(\frac{-1}{\sqrt{3}}\right)=-\tan^{-1}\left(\frac{1}{\sqrt{3}}\right)=\frac{-\pi}{6}$. So $z=\cis\left(\frac{-\pi}{6}\right)$. Then $z^{100}=\cis\left(\frac{-100\pi}{6}\right)=\cis\left(\frac{-50\pi}{3}\right)$. We have $2\pi=\frac{6\pi}{3}$, so $\frac{48\pi}{3}=8\times 2\pi$, so $\cis\left(\frac{-50\pi}{3}\right)=\cis\left(\frac{-2\pi}{3}\right)$. We can leave this as our final answer in polar form, or revert to cartesian form: $\frac{-1}{2}+\frac{\sqrt{3}}{2}j$.

\item Again, we use de Moivre's Theorem. $|-2+j|=\sqrt{4+1}=\sqrt{5}$, and $\tan(\arg(-2+j))=\frac{1}{-2}$; we are in the top right quadrant, so $\arg(-2+j)=\pi+\tan^{-1}\left(\frac{-1}{2}\right)\approx 2.678$. So we have
\[-2+j = \sqrt{5}\cis(2.678) = \sqrt{5}\cis(8.961) = \sqrt{5}\cis(15.244)\]
by adding $2\pi$ and $4\pi$ to the argument. Now, $\sqrt[3]{\sqrt{5}}=\sqrt[6]{5}\approx 1.308$ and dividing the three arguments by 3 gives $0.893$, $2.987$, and $5.081$ respectively. So the cube roots are:
\[1.308\cis(0.893),\qquad 1.308\cis(2.987),\qquad 1.308\cis(5.081).\]
We could convert these back to cartesian form if we wished.
\end{enumerate}






\end{document}