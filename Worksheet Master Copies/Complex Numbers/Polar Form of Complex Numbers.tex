\documentclass{article}

\usepackage[left=2cm,right=2cm, top=2cm, bottom = 2cm]{geometry}
\usepackage{amsfonts}
%%%\usepackage{array}

\usepackage{tikz}

\pagestyle{empty}

%%%\setlength{\tabcolsep}{1.8cm}
%%%\renewcommand{\arraystretch}{2.5}

%%%\makeatletter
%%%\newcommand{\thickhline}{%
%%%    \noalign {\ifnum 0=`}\fi \hrule height 2pt
%%%    \futurelet \reserved@a \@xhline
%%%}
%%%\newcolumntype{!}{@{\hskip\tabcolsep\vrule width 2pt\hskip\tabcolsep}}
%%%\makeatother

\newcommand{\cis}{\,\mathrm{cis}}

\begin{document}

\title{Polar Form of Complex Numbers}
\date{}

\maketitle
\thispagestyle{empty}

\Large

\textbf{\underline{Objective: To understand the polar form of complex numbers}}

\textbf{\underline{and its uses. To be able to convert complex numbers between}}

\textbf{\underline{polar and cartesian forms.}}

\vspace{5mm}


{\bf Recap of previous material:}

\vspace{5mm}

\begin{enumerate}
\item Let $z=-1.3-9.4j$. Write down $\mathrm{Re}(z)$, $\mathrm{Im}(z)$, and $\bar{z}$.
\item $(2-3j)+(4j-7)=$
\item $j(1+5j)=$
\item $(1-2j)(1+2j)=$
\item $\frac{4-j}{1+4j}=$
\end{enumerate}


\clearpage

{\bf Warm-up:}

\vspace{5mm}

We will explore a link between complex numbers and their complex conjugates, and polar coordinates.

\begin{enumerate}
\item Let $z=2-3j$.
	\begin{enumerate}
	\item Calculate $\sqrt{z\bar{z}}$.
	\item We can represent $z$ graphically by the point $(2,-3)$ in the $(x,y)$-plane. Convert this point into polar coordinates.
	\item Compare the two answers above.
	\end{enumerate}
\item Let $z=a+bj$ for some real numbers $a$ and $b$.
	\begin{enumerate}
	\item Calculate $z\bar{z}$ in terms of $a$ and $b$.
	\item Is $z\bar{z}$ real, imaginary, or complex? What about $\sqrt{z\bar{z}}$?
	\item Represent $z$ by the point $(a,b)$ in the $(x,y)$-plane. Convert this point into polar coordinates.
	\item Compare the answers above.
	\end{enumerate}
\end{enumerate}

\clearpage



\textbf{Theory---Cartesian and Polar Forms of Complex Numbers:}

\vspace{5mm}

We have seen that a complex number $z$ has the form $a+bj$ for two real numbers $a$ and $b$---called the \textbf{real part} and \textbf{imaginary part} respectively. We can represent this as the point $(a,b)$ in the $(x,y)$-plane. Because we are treating the real and imaginary parts of $z$ as cartesian coordinates of a point, we call $a+bj$ the \textbf{cartesian form} of $z$. When we plot a point in the plane to represent a complex number, we call that diagram the \textbf{complex plane} or an \textbf{Argand diagram}.\medskip

Suppose a complex number $z$ is represented in the Argand diagram by a point whose polar coordinates are $(r,\theta)$. Find the real and imaginary parts of $z$.

\vfill

So we see that
\[z=r\cos(\theta)+rj\sin(\theta)=r\left[\cos(\theta)+j\sin(\theta)\right].\]

We sometimes write $\cis(\theta)$ as shorthand for $\cos(\theta)+j\sin(\theta)$ (``cis'' stands for ``$\cos+i\sin$'', using $i$ in place of $j$). So then $z=r\cis(\theta)$. This is called the \textbf{polar form} of $z$ (whether written in terms of $\cis$ or longhand with $\sin$ and $\cos$).

We saw in the second warm-up question that $z\bar{z}$ is always a non-negative real number, so has a non-negative real square root, which is precisely the $r$ coordinate of $z$ in polar coordinates. This number is called the \textbf{modulus} of $z$ and written $|z|$. The $\theta$-coordinate of $z$ is called the \textbf{argument} of $z$ and written $\arg(z)$. We have for $z=a+bj$:
\[|z|=\sqrt{z\bar{z}}=\sqrt{a^2+b^2}\qquad \tan(\arg(z))=\frac{b}{a}.\]

As always with polar coordinates, you have to think about the quadrant the point is in to figure out whether $\arg(z)=\tan^{-1}\left(\frac{b}{a}\right)$ or $\arg(z)=\pi+\tan^{-1}\left(\frac{b}{a}\right)$.











\clearpage


\textbf{Practice:}

\vspace{5mm}

We shall practise converting between cartesian and polar forms, then explore why polar form is useful.

\begin{enumerate}
\item Let $z=1-j$.
	\begin{enumerate}
	\item Calculate the modulus $|z|$.
	\item Calculate the argument $\arg(z)$.
	\item Hence express $z$ in polar form.
	\end{enumerate}
\item Let $w=2\cis\left(\frac{\pi}{3}\right)$.
	\begin{enumerate}
	\item Write down $|w|$.
	\item Write down $\arg(w)$.
	\item Calculate $\mathrm{Re}(w)$.
	\item Calculate $\mathrm{Im}(w)$.
	\item Hence express $w$ in cartesian form.
	\end{enumerate}
\item Take $z$ and $w$ from the previous two questions.
	\begin{enumerate}
	\item Working in cartesian form, calculate the product $zw$.
	\item Calculate $|zw|$. Compare with $|z|$ and $|w|$.
	\item Calculate $\arg(zw)$. Compare with $\arg(z)$ and $\arg(w)$.
	\item Now calculate $\frac{z}{w}$, again working in cartesian form.
	\item Calculate $\left|\frac{z}{w}\right|$. Compare with $|z|$ and $|w|$.
	\item Calculate $\arg\left(\frac{z}{w}\right)$. Compare with $\arg(z)$ and $\arg(w)$.
	\end{enumerate}
\end{enumerate}





\clearpage





\textbf{Theory---Multiplication in Polar Form:}\bigskip

Let $z=r\cis(\theta)$ and $w=s\,\cis(\phi)$ be two complex numbers written in polar form. Multiply them together and collect real and imaginary terms to find an expression for $zw$. Hence find $|zw|$.

\vfill



Use the compound angle formulae to expand $rs\,\cis(\theta+\phi)$. Compare with the above.

\vfill


So we see that if we multiply two complex numbers in polar form, we can use the compound angle formulae to simplify the answer to a single complex number in polar form, whose modulus is $|z|\times |w|$ and whose argument is $\arg(z)+\arg(w)$. This leads to the following simple rule for multiplying complex numbers:

\begin{center}
\textbf{\underline{M}ultiply the \underline{M}oduli and \underline{A}dd the \underline{A}rguments.}
\end{center}

As an equation,
\[(r\cis(\theta))(s\,\cis(\phi))=rs\,\cis(\theta+\phi).\]



\clearpage


\textbf{Theory---Dividing in Polar Form:}

We shall now apply the above method for multiplication to division of complex numbers in polar form.

Suppose we have two complex numbers, $z$ and $w$, and want to compute $\frac{z}{w}$. Let $y=\frac{z}{w}$; then $y$ is defined by the property that $wy=z$. Since to multiply two complex numbers we \underline{M}ultiply the \underline{M}oduli and \underline{A}dd the \underline{A}rguments, we must have
\[|w|\times |y|=|z|\qquad\mbox{ and }\qquad \arg(w)+\arg(y)=\arg(z).\]

Rearranging these, we see that
\[|y|=\frac{|z|}{|w|}\qquad\mbox{ and }\qquad \arg(y)=\arg(z)-\arg(w).\]

So to divide complex numbers, we \textbf{divide the moduli and subtract the arguments.} As an equation,
\[\frac{r\cis(\theta)}{s\,\cis(\phi)}=\frac{r}{s}\cis(\theta-\phi).\]


In particular, if we want to find $\frac{1}{z}$ for some complex number $z$, we use that $|1|=1$ and $\arg(1)=0$, so
\[\left|\frac{1}{z}\right|=\frac{1}{|z|}\qquad\mbox{ and }\qquad \arg\left(\frac{1}{z}\right)=-\arg(z).\]

\vspace{5mm}

Next time we shall see \textit{de Moivre's Theorem}---a simple but powerful use of this multiplication/division rule in polar form, to find powers and roots of complex numbers quickly and easily.






\clearpage


{\bf Key Points to Remember}

\vspace{5mm}

\begin{enumerate}
\item The \textbf{cartesian form} of a complex number $z$ is its normal expression as $a+bj$ for some real numbers $a$ and $b$.
\item The point $z=a+bj$ can be represented on the \textbf{complex plane} (or \textbf{Argand diagram}) by plotting it as the point $(a,b)$.
\item If we express $(a,b)$ in polar coordinates $(r,\theta)$, then $r$ is called the \textbf{modulus} of $z$, written $|z|$, and $\theta$ the \textbf{argument} of $z$, written $\arg(z)$.
\item The \textbf{polar form} of $z$ is the expression $|z|(\cos[\arg(z)]+j\sin[\arg(z)])$, often abbreviated to $|z|\cis(\arg(z))$.
\item To multiply complex numbers in polar form, we use the rule: \textbf{\underline{M}ultiply the \underline{M}oduli and \underline{A}dd the \underline{A}rguments}:
	\[|zw|=|z|\times |w|\qquad \mbox{ and }\qquad\arg(zw)=\arg(z)+\arg(w).\]
\item To divide complex numbers in polar form, we use this in reverse, we \textbf{divide the moduli and subtract the arguments}:
	\[\left|\frac{z}{w}\right|=\frac{|z|}{|w|}\qquad\mbox{ and }\qquad \arg\left(\frac{z}{w}\right)=\arg(z)-\arg(w).\]
\item Cartesian form is (much) better to use when adding or subtracting complex numbers. Though multiplication and division aren't too hard in cartesian form, they're very easy in polar form.
\end{enumerate}





\end{document}