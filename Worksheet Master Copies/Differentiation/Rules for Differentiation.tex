\documentclass{article}

\usepackage[left=2cm,right=2cm, top=2cm, bottom = 2cm]{geometry}
\usepackage{amsfonts}
%%%\usepackage{array}

\usepackage{amsmath}
\usepackage{xcolor}

\usepackage{tikz}
\usepackage{subfigure}

\pagestyle{empty}

\setlength{\tabcolsep}{15pt}
%%%\renewcommand{\arraystretch}{2.5}

%%%\makeatletter
%%%\newcommand{\thickhline}{%
%%%    \noalign {\ifnum 0=`}\fi \hrule height 2pt
%%%    \futurelet \reserved@a \@xhline
%%%}
%%%\newcolumntype{!}{@{\hskip\tabcolsep\vrule width 2pt\hskip\tabcolsep}}
%%%\makeatother

\newcommand{\deriv}[2]{\frac{\mathrm{d}#1}{\mathrm{d}#2}}




\begin{document}

\title{Rules for Differentiation.}
\date{}

\maketitle
\thispagestyle{empty}

\Large

\textbf{\underline{Objective: To understand rules for differentiating various functions.}}





\vspace{5mm}


\textbf{Recap: Monotonicity and Extrema:}

\vspace{5mm}

Let $f(t)=4t-t^2$.
\begin{enumerate}
	\item Find the derivative of $f(t)$ with respect to $t$.
	\item Solve $f'(t)=0$.
	\item By examining whether $f'(t)$ is positive or negative, determine for which $t$ values $f(t)$ is increasing, and for which it is decreasing.
	\item Hence determine whether your solution to part 2 is a maximum or minimum.
\end{enumerate}



\clearpage


\textbf{Warm-up:}

\vspace{5mm}

We will prove that
\[\deriv{x^n}{x}=nx^{n-1}\]
for $n$ a positive integer. To do this, we will need to evaluate
\[\lim_{h\to 0} \frac{(x+h)^n-x^n}{h}.\]

\begin{enumerate}
	\item Consider
		\[(x+h)^n=(x+h)(x+h)\hdots(x+h).\]
		Without knowing $n$, we cannot multiply this out---we do not know how many brackets there are. However, we can say (in terms of $n$) what the term with $x^n$ in will be. What is it?
	\item When multiplying out, we pick one item from each bracket, and do this in all possible ways. How many ways are there of picking out $h$ from one bracket and $x$ from all the othes? Therefore, what is the term involving $hx^{n-1}$ in the expansion?
	\item Every other term in the expansion must involve $h$ at least twice---\textit{i.e.}, must be a multiple of $h^2$. So we can write
		\[(x+h)^n=x^n+nhx^{n-1}+h^2F,\]
		where $F$ is some unknown combination of $x$ and $h$. Using this, show that
		\[\lim_{h\to 0} \frac{(x+h)^n-x^n}{h}=nx^{n-1}.\]
\end{enumerate}








\clearpage


{\bf Theory: Linearity, and the Chain and Product Rules:}

\vspace{5mm}

So far we have seen the definition of the derivative, and some uses - determining whether a function is increasing or decreasing, and finding maxima and minima. But we don't have any techniques for computing the derivative of a function, apart from directly applying the definition. Now we will study three extremely useful rules for differentiation.\bigskip

Suppose $f(t)$ and $g(t)$ are functions such that $f'(t)$ and $g'(t)$ are known. Find the derivative of $af(t)+bg(t)$, where $a$ and $b$ are constants.

\vfill


Suppose $f$ and $g$ are as above. Find the derivative of the product, $p(t)=f(t)g(t)$.

\vfill


Suppose $f$ and $g$ are as above. Find the derivative of the composition $c(t)=f(g(t))$.

\vfill



\clearpage



\textbf{Examples of Computing Derivatives:}

\vspace{5mm}


Let $y=x^{-n}$, where $n$ is a positive integer. Compute $\deriv{y}{x}$.

\vfill


Let $y=x^{m/n}$, where $m$ and $n$ are positive integers. Compute $\deriv{y}{x}$.

\vfill

Suppose that $f(x)$ is an invertible function and $\deriv{f}{x}$ is known. Find $\deriv{f^{-1}}{x}$ by considering $g(x)=f(f^{-1}(x))$.



\clearpage








\textbf{Theory: Derivatives of Trigonometric and Exponential Functions:}

\vspace{5mm}


Differentiate $\sin(x)$ and $\cos(x)$, making use of the small-angle approximations:
\[\sin(h)\approx h\qquad\mbox{ and } \qquad\cos(h)\approx 1-\frac{h^2}{2}\]
for $h$ close to 0.


\vfill


Let $b>0$, $b\neq 1$. Differentiate $b^x$ with respect to $x$.

\vfill

We will need the following Python code, which can be executed at \begin{verbatim} https://www.onlinegdb.com/online_python_interpreter \end{verbatim}

\begin{verbatim}

def f(base):
    print("base = %f" % base)
    for i in range(7):
        h=10**(-i)
        derivative_at_zero = (base**h - 1) / h
        print("h=%f, f'(0)=%f" % (h,derivative_at_zero))
    print("\n")
        
f(2)
f(3)
f(2.5)
f(2.7)
f(2.71828183)

\end{verbatim}



\clearpage









\textbf{Further Examples of Computing Derivatives:}

\vspace{5mm}


From a previous page, we have that
\[\deriv{x^n}{x}=nx^{n-1}\]
for any rational number $n$.

\bigskip

Differentiate $\log_e(x)$ with respect to $x$.

\vfill


Differentiate $4x^2+3x-2$ with respect to $x$.

\vfill

Differentiate $(2x-1)(x+1)$ with respect to $x$.

\vfill


Differentiate $(12t^2+2t-3)^7$ with respect to $x$.

\vfill




\clearpage









\textbf{Key Derivatives to Know:}

\bigskip

\[\deriv{x^n}{x}=nx^{n-1}\]

\vfill

\[\deriv{\sin(x)}{x}=\cos(x)\]

\vfill

\[\deriv{\cos(x)}{x}=-\sin(x)\]

\vfill

\[\deriv{e^x}{x}=e^x\]

\vfill

\[\deriv{\log_e(x)}{x}=\frac{1}{x}\]





\clearpage





\textbf{Practice:}

\vspace{5mm}

The chain rule:
\[\mbox{if } h(t)=f(g(t)),\mbox{ then } h'(t)=g'(t)f'(g(t)).\]

The product rule:
\[\mbox{if } h(t)=f(t)g(t),\mbox{ then } h'(t)=f(t)g'(t)+f'(t)g(t).\]

Linearity, for constants $a$ and $b$:
\[\deriv{\,(af(t)+bg(t))}{t}=a\deriv{f(t)}{t}+b\deriv{g(t)}{t}.\]



\begin{enumerate}
	\item Let $f(t)=2t+1$ and $g(t)=4t-6$.
		\begin{enumerate}
			\item Compute $f'(t)$.
			\item Compute $g'(t)$.
			\item Use the product rule to differentiate $h(t)=f(t)g(t)$.
			\item Use the chain rule to differentiate $i(t)=f(g(t))$.
		\end{enumerate}
	\item Differentiate $(x^2+2x+3)^{10}$ with respect to $x$ (hint: apply the chain rule with $f(x)=x^{10}$, $g(x)=x^2+2x+3$).
	\item Differentiate $(7x-3)^{500}$ with respect to $x$ (hint: chain rule).
	\item Differentiate $(4x-1)(2x+1)$ with respect to $x$ (hint: apply the product rule with $f(x)=4x-1$, $g(x)=2x+1$).
	\item Differentiate $(13x^2-2)(4x+6)$ with respect to $x$ (hint: product rule).
	\item Differentiate $(4x^2+6x-2)^5(x^3+1)$ with respect to $x$ (hint: use both the product and chain rules).
	\item Let $y=\sin(4x+1)$. Find $\deriv{y}{x}$.
	\item Let $y=2\sin(x)\cos(x)$.
		\begin{enumerate}
			\item Find $\deriv{y}{x}$ by the product rule.
			\item By the compound angle formula for sine, $y=\sin(2x)$. Hence find $\deriv{y}{x}$ by the chain rule.
			\item Show that your two answers are the same.
		\end{enumerate}
	\item Differentiate $\tan(x)$ with respect to $x$ (hint: write $\tan(x)=\sin(x)(\cos(x))^{-1}$ and apply the product and chain rules).
\end{enumerate}



\clearpage




\textbf{Application:}

\vspace{5mm}


Suppose a 10V DC supply is used to drive current through a variable resistor. The resistance is varied according to $R=\frac{1000}{(t^2-10t+26)(t+1)}$.



\begin{enumerate}
	\item Find the current $I$ through the resistor at time $t$.
	\item By differentiating, find the rate of change of current with respect to time, $\deriv{I}{t}$.
	\item Solve $\deriv{I}{t}=0$.
	\item By considering whether $\deriv{I}{t}$ is positive or negative to either side of the solutions from the previous part, determine whether these points are maxima, minima, or points of inflexion.
	\item Find the maximum and minimum current through the resistor at any time between $t=0$ and $t=10$, and the times these currents occur.
\end{enumerate}


\clearpage


{\bf Key Points to Remember:}

\vspace{5mm}

\begin{enumerate}
	\item Linearity:
		\[\deriv{ax+by}{t}=a\deriv{x}{t}+b\deriv{y}{t}.\]
		for any constants $a$ and $b$.
	\item The product rule:
		\[\deriv{xy}{t}=x\deriv{y}{t}+y\deriv{x}{t}.\]
	\item The chain rule:
		\[\mbox{if } h(t)=f(g(t)),\mbox{ then } h'(t)=f'(g(t))g'(t).\]
	\item \[\deriv{x^n}{x}=nx^{n-1}.\]
	\item \[\deriv{\sin(x)}{x}=\cos(x).\]
	\item \[\deriv{\cos(x)}{x}=-\sin(x).\]
\end{enumerate}









\end{document}