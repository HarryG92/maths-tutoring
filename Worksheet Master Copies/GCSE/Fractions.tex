\documentclass{article}

\usepackage[left=2cm,right=2cm, top=2cm, bottom = 2cm]{geometry}
\usepackage{amsfonts}
\usepackage{array}

\setlength{\tabcolsep}{1.8cm}
\renewcommand{\arraystretch}{2.5}

\makeatletter
\newcommand{\thickhline}{%
    \noalign {\ifnum 0=`}\fi \hrule height 2pt
    \futurelet \reserved@a \@xhline
}
\newcolumntype{!}{@{\hskip\tabcolsep\vrule width 2pt\hskip\tabcolsep}}
\makeatother

\pagestyle{empty}

\begin{document}

\title{Fractions}
\date{}

\maketitle
\thispagestyle{empty}

\Large

\textbf{\underline{Objective: To be confident doing arithmetic with fractions.}}

\vspace{5mm}

\textbf{Recap of Previous Material:}

\vspace{5mm}

\begin{enumerate}
	\item Express both parts of $1:5$ as fractions.
	\item Express both parts of $7:11$ as fractions.
	\item Write $\frac{5}{9}$ as a ratio.
	\item Write $\frac{14}{57}$ as a ratio.
\end{enumerate}

\clearpage


\textbf{Warm-up:}

\begin{enumerate}
	\item $\frac{1}{5}+\frac{1}{5}=$
	\item $\frac{4}{7}+\frac{2}{7}=$
	\item $\frac{3}{9}-\frac{1}{9}=$
	\item $\frac{12}{4}-\frac{1}{4}=$
	\item $\frac{1}{3}\times \frac{1}{2}=$
	\item $\frac{2}{3}\times \frac{5}{4}=$
\end{enumerate}




\clearpage


{\bf Theory --- Adding and Subtracting Fractions:}

\vspace{5mm}


$\frac{1}{12} + \frac{3}{4}=$

\vfill

$\frac{7}{6}+\frac{4}{15}=$

\vfill

$\frac{8}{9}-\frac{1}{3}=$

\vfill

$\frac{5}{12}-\frac{9}{10}=$





\clearpage



{\bf Practice:}

\vspace{5mm}

\begin{enumerate}
	\item $\frac{1}{3}+\frac{12}{21}=$
	\item $\frac{7}{6}+\frac{2}{11}=$
	\item $\frac{4}{9}-\frac{2}{3}=$
	\item $\frac{13}{16}-\frac{29}{40}=$
\end{enumerate}




\clearpage





\textbf{Theory --- Multiplying and Dividing Fractions:}

\vspace{5mm}

$\frac{1}{2}\times\frac{5}{4}=$

\vfill

$\frac{2}{7}\times\frac{11}{13}=$

\vfill

$1\div\frac{2}{3}=$

\vfill

$\frac{3}{4}\div \frac{7}{2}=$



\clearpage








\textbf{Practice:}

\vspace{5mm}



\begin{enumerate}
	\item $\frac{6}{17}\times\frac{1}{3}=$
	\item $\frac{5}{12}\times \frac{4}{9}=$
	\item $\frac{1}{8}\times \frac{6}{19}=$
	\item $\frac{3}{7}\div\frac{1}{8}=$
	\item $\frac{3}{8}\div\frac{16}{25}=$
	\item $\frac{5}{9}\div\frac{11}{6}=$
\end{enumerate}












\end{document}