\documentclass{article}

\usepackage[left=2cm,right=2cm, top=2cm, bottom = 2cm]{geometry}
\usepackage{amsfonts}
%%%\usepackage{array}

\usepackage{amsmath}
\usepackage{xcolor}

\usepackage{tikz}
\usepackage{subfigure}



\pagestyle{empty}

\setlength{\tabcolsep}{15pt}
%%%\renewcommand{\arraystretch}{2.5}

%%%\makeatletter
%%%\newcommand{\thickhline}{%
%%%    \noalign {\ifnum 0=`}\fi \hrule height 2pt
%%%    \futurelet \reserved@a \@xhline
%%%}
%%%\newcolumntype{!}{@{\hskip\tabcolsep\vrule width 2pt\hskip\tabcolsep}}
%%%\makeatother

\newcommand{\deriv}[3][]{\frac{\mathrm{d}^{#1}#2}{\mathrm{d}#3^{#1}}}
\newcommand{\diff}{\;\mathrm{d}}


\begin{document}

\title{Integration by Change of Variables}
\date{}

\maketitle
\thispagestyle{empty}

\Large

\textbf{\underline{Objective: To practise identifying which method to use to integrate}}

\textbf{\underline{a function, and applying the different techniques we have learnt.}}







\vspace{5mm}



\textbf{A Checklist for Integration:}\bigskip

\begin{enumerate}
	\item Is the integrand continuous? If not, will need to split the domain of integration and miss out a strip or width $\epsilon$ around the discontinuity, then take the limit as $\epsilon\to 0$.
	\item Is it a standard integral, such as $\sin(x)$, $\cos(x)$, $x^n$, $e^x$, $\sinh(x)$, or $\cosh(x)$? Possibly with constants multiplying/added to the variable.
	\item Can I rewrite the integrand in some more convenient form? For instance, factorising or expanding out brackets, rewriting roots as fractional powers, using trig identities, etc.
	\item Is it a chain rule integral? That is, can I split the integrand into a function $g(f(x))$, times the derivative of the inside function, $f'(x)$? Then this came from the chain rule and so an antiderivative is $f(g(x))$. Might have to multiply by a constant at the end to make it work out exactly.
	\item Is there a subsitution I could make that might simplify the integral? Trig or hyperbolic substitutions are often useful when there are expressions involving squares, because of all the trig identities with squared functions in.
	\item Can I split the integrand as a product of two functions, $u$ and $\deriv{v}{x}$, and use integration by parts?
\end{enumerate}

\clearpage



\textbf{Practice:}\bigskip


\begin{enumerate}
	\item \[\int (7x^3+2e^{3x}-9\cos(x))\diff x=\]
	\item \[\int_{-2}^2 \frac{1}{\sqrt{x}}\diff x=\]
	\item \[\int t^3(t^2+t^{1/2})(t^7-3t^{-3/2})\diff t=\]
	\item \[\int (2\sin(t)\cos(t)\sin(2t)+cos^2(2t) )\diff t=\]
	\item \[\int t^2(t^3-4)^9\diff t=\]
	\item \[\int y^{-1/2}\cos\left(\sqrt{y}\right)\diff y=\]
	\item \[\int \left(y^2-\frac{2}{3}y\right)e^{y^3-y^2}\diff y=\]
	\item \[\int \tan(\theta)\diff \theta=\]
	\item \[\int\frac{1}{\sqrt{1+z^2}}\diff z=\]
	\item \[\int\frac{\sqrt{z^3-1}}{z}\diff z=\]
	\item \[\int ze^{z}\diff z=\]
	\item \[\int u\cos(2u)\diff u=\]
\end{enumerate}

\clearpage


Some standard useful tricks for particular integrals:

\begin{enumerate}
	\item To integrate $\ln(x)$, write as $1.\ln(x)$ and use parts, with $u=\ln(x)$, $\deriv{v}{x}=1$.
	\item To integrate $\sin^2(x)$ or $\cos^2(x)$, use the double angle formulae for cosine to rewrite in terms of $\cos(2x)$.
	\item To integrate products of exponentials and trig functions, use Euler's Equation to rewrite in terms of real or imaginary parts of complex exponentials.
\end{enumerate}





\end{document}