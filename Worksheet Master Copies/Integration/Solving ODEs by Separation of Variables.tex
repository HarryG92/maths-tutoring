\documentclass{article}

\usepackage[left=2cm,right=2cm, top=2cm, bottom = 2cm]{geometry}
\usepackage{amsfonts}
%%%\usepackage{array}

\usepackage{amsmath}
\usepackage{xcolor}

\usepackage{tikz}
\usepackage{subfigure}



\pagestyle{empty}

\setlength{\tabcolsep}{15pt}
%%%\renewcommand{\arraystretch}{2.5}

%%%\makeatletter
%%%\newcommand{\thickhline}{%
%%%    \noalign {\ifnum 0=`}\fi \hrule height 2pt
%%%    \futurelet \reserved@a \@xhline
%%%}
%%%\newcolumntype{!}{@{\hskip\tabcolsep\vrule width 2pt\hskip\tabcolsep}}
%%%\makeatother

\newcommand{\deriv}[3][]{\frac{\mathrm{d}^{#1}#2}{\mathrm{d}#3^{#1}}}
\newcommand{\diff}{\;\mathrm{d}}


\begin{document}

\title{Solving ODEs: Separation of Variables}
\date{}

\maketitle
\thispagestyle{empty}

\Large

\textbf{\underline{Objective: To be able to identify and solve separable, first-order}}

\textbf{\underline{ODEs.}}







\vspace{5mm}







\textbf{Warm-up: Radioactive Decay:}\bigskip


Consider a sample of a radioactive element, and let $N$ be the number of atoms in the sample that have not yet decayed. Each atom decays at some point \textit{at random}, each atom having the same probability $p$ of decaying within a given second (or other short time interval). The number of atoms you expect to decay in a given second is equal to the probability of any given atom decaying, $p$, times the number of atoms, $N$.

Therefore in a small period of time, you expect $N$ to decrease by $pN$. In other words, the rate at which $N$ changes is equal to minus $pN$, so $N$ obeys the \textbf{first-order ordinary differential equation}
\[\deriv{N}{t}=-p N.\]

We shall solve this equation, and study its solution somewhat.

\begin{enumerate}
	\item Use the substitution formula for integration (in reverse!) to show that
		\[\int \frac{1}{N}\deriv{N}{t}\diff t=\int\frac{1}{N}\diff N\]
		and hence evaluate this integral.
	\item Divide both sides of the ODE by $N$ and integrate with respect to $t$ to show that
		\[\ln(N)=-pt+c,\]
		where $c$ is an unknown constant.
	\item Let $N_0$ be the number of undecayed atoms at time $t=0$. Hence find $c$.
	\item Hence show that
		\[N=N_0 e^{-pt}\]
	\item Find an expression in terms of $p$ for the time after which eactly half of the original number of atoms has decayed.
\end{enumerate}




\clearpage










\textbf{Theory: Separation of Variables:}

\bigskip

We shall generalise the idea we employed in the warm-up to develop a general technique called \textbf{separation of variables} for solving certain types of first-order ODE.

A first-order ODE is called \textbf{separable} if it can be written in the form
\[\deriv{x}{t} = f(x)g(t)\]
for some functions $f$ and $g$. Then we can divide through by $f$ and integrate with respect to $t$ to get
\begin{equation}
	\int\frac{1}{f(x)}\deriv{x}{t}\diff t=\int g(t)\diff t.\tag{$\star$}
\end{equation}
Now we use the substitution formula for integrals, but in reverse from how we usually use it. Imagine we wanted to find
\[\int\frac{1}{f(x)}\diff x;\]
we could introduce a new variable $t$, vary $x$ according to some function of $t$, and have by substitution:
\[\int \frac{1}{f(x)}\diff x = \int \frac{1}{f(x)}\deriv{x}{t}\diff t.\]
But the right-hand side of this is precisely the left-hand side of equation $(\star)$! Therefore we can put this into equation $(\star)$ to get
\[\int\frac{1}{f(x)}\diff x = \int g(t)\diff t.\]
As long as we can integrate $\frac{1}{f(x)}$ and $g$, then, we can find an equation linking $x$ and $t$, which we can then hope to rearrange to get $x$ as a function of $t$ and thereby have a solution to our original ODE.\medskip


Solve
\[\deriv{x}{t} = x^2e^{-t}\]
with the \textbf{initial condition} $x(0)=1$.

\vfill


Find the general solution to
\[\deriv{y}{x} = x\tan(y).\]












\clearpage






\textbf{Practice:}\bigskip



\begin{enumerate}
	\item Find the particular solution to
		\[\deriv{y}{x}=y\sin(x)\]
		satisfying the boundary condition that $y=2$ when $x=\frac{\pi}{2}$.
	\item Find the general solution to
		\[\deriv{y}{t}=\frac{t e^{t^2}}{y}.\]
	\item A capacitor discharges from voltage $V_0$ at time 0 through a resistor.
		\begin{enumerate}
			\item Using the equations $Q=CV_\mathrm{C}$ (charge is capacitance times capacitor voltage) and $V_\mathrm{R}=IR$ (Ohm's Law: resistor voltage is current times resistance), set up and solve a differential equation describing the voltage $V_\mathrm{C}$ on the capacitor.
			\item Find (in terms of $R$ and $C$) the time after which the voltage has decayed to $\frac{V_0}{e}$. This is called the \textbf{time constant} of the RC circuit. Note that it does not depend on the value of $V_0$!
			\item Assume that $C=200\mu\mathrm{F}$, $R=1\mathrm{k\Omega}$, and $V_0=10\mathrm{V}$. After how long will the voltage across the capacitor be 1V?
		\end{enumerate}
\end{enumerate}











\clearpage





\textbf{Application: Population Growth:}\bigskip


\begin{enumerate}
	\item A simple model for the growth of a population (of humans, animals, plants, bacteria, whatever...) says that the population grows at a rate proportional to its current size; so if $P$ is the number of organisms in the population, then
		\[\deriv{P}{t}=\lambda P,\]
		where $\lambda$ is the growth rate, which measures how many more births than deaths there are. Solve this differential equation subject to the condition that the initial population is $P_0$.
	\item In your solution, what happens as $t\to\infty$ (assuming $\lambda$ is positive)? Does this seem realistic?
	\item A more sophisticated model of population growth assumes that the habitat has a \textbf{carrying capacity} $P_\mathrm{max}$: a maximum population that it can support. If $P$ is small, we should have roughly the growth we had before, but when $P$ gets close to $P_\mathrm{max}$, the growth rate should drop towards 0. The following ODE has these properties:
		\[\deriv{P}{t}=\lambda P\left(1-\frac{P}{P_\mathrm{max}}\right).\]
		Consider the value of $\left(1-\frac{P}{P_\mathrm{max}}\right)$ for different values of $P$---much smaller than, slightly smaller than, equal to, and bigger than the carrying capacity---to convince yourself that this ODE has the properties we want.
	\item By separating variables in this ODE, show that
		\[\int \frac{P_\mathrm{max}}{P(P_\mathrm{max}-P)}\diff P = \int \lambda \diff t.\]
	\item Show that
		\[\frac{1}{P}+\frac{1}{P_\mathrm{max}-P}=\frac{P_\mathrm{max}}{P(P_\mathrm{max}-P)}.\]
	\item Hence solve the ODE (with initial population $P_0$) to show that
		\[\frac{P}{P_\mathrm{max}-P}=\frac{P_0}{P_\mathrm{max}-P_0} e^{\lambda t}.\]
	\item Multiply both sides of this equation by $(P_\mathrm{max}-P)$ and rearrange to show that
		\[P=\frac{P_\mathrm{max} P_0 }{P_0  +(P_\mathrm{max}-P_0)e^{-\lambda t}}.\]
	\item What happens to the population as $t\to \infty$?
\end{enumerate}















\clearpage




{\bf Key Points to Remember:}

\vspace{5mm}

\begin{enumerate}
	\item A first-order ODE is called \textbf{separable} if it can be written in the form
		\[\deriv{x}{t} = f(x)g(t)\]
		for some functions $f$ and $g$.
	\item Given a separable ODE as above, we can solve by dividing through by $f(x)$ and integrating with respect to $t$ to obtain
		\[\int\frac{1}{f(x)}\diff x = \int\frac{1}{f(x)}\deriv{x}{t}\diff t = \int g(t)\diff t +c.\]
		This process is called \textbf{separation of variables}.
	\item Initial conditions can be used to find the value of the constant of integration.
	\item A very common type of separable equation has the simple form
		\[\deriv{x}{t}=\lambda x\]
		for some constant $\lambda$. The solution to this is
		\[x=x_0e^{\lambda t},\]
		where $x_0$ is the value of $x$ at $t=0$. If $\lambda>0$, the solution grows exponentially (\textit{e.g.}, population growth when the population is far smaller than the habitat can support), if $\lambda<0$, the solution decays exponentially towards 0 (\textit{e.g.}, voltage on a discharging capacitor, radioactive decay, metabolism of a drug in the body, attenuation of radiation as it passes through a material).
	\item In exponential growth/decay, $e^{\lambda t}$, the constant $\lambda$ is called the growth rate (if positive) or the decay rate (if negative). If $\lambda<0$, then $\frac{-1}{\lambda}$ is called the \textbf{time constant}; it is the time taken for the function to decay to $\frac{1}{e}$ of its prior value.
\end{enumerate}









\end{document}