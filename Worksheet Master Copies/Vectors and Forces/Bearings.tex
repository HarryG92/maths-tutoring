\documentclass{article}

\usepackage[left=2cm,right=2cm, top=2cm, bottom = 2cm]{geometry}
\usepackage{amsfonts}
%%%\usepackage{array}

\usepackage{tikz}

\pagestyle{empty}

\setlength{\tabcolsep}{15pt}
%%%\renewcommand{\arraystretch}{2.5}

%%%\makeatletter
%%%\newcommand{\thickhline}{%
%%%    \noalign {\ifnum 0=`}\fi \hrule height 2pt
%%%    \futurelet \reserved@a \@xhline
%%%}
%%%\newcolumntype{!}{@{\hskip\tabcolsep\vrule width 2pt\hskip\tabcolsep}}
%%%\makeatother

\newcommand{\ihat}{\hat{\i}}
\newcommand{\jhat}{\hat{\j}}
\newcommand{\khat}{\hat{k}}
\let\rarrow\overrightarrow

\begin{document}

\title{Bearings and Moments}
\date{}

\maketitle
\thispagestyle{empty}

\Large


\section{Bearings}


\begin{enumerate}
	\item A ship leaves port at $O$ and sails 12 nautical miles at a bearing of $115^\circ$, to $A$. It then sails 5 nautical miles at a bearing of $242^\circ$ to $B$, before returning directly to $O$.
		\begin{enumerate}
			\item Find the position vector of $A$ relative to $O$.
			\item Find the position vector of $B$ relative to $O$.
			\item Find the bearing taken on the last leg of the journey.
		\end{enumerate}
	\item Two ships set sail, one from a point with position vector $-3\ihat-4\jhat$, and the other from $9\ihat-6\jhat$. They plan to travel in straight lines and rendezvous at the origin. Find the bearing each ship must move on, and the distance it must travel.
\end{enumerate}

\clearpage


\section{Moments}

\begin{enumerate}
	\item A seesaw of length 2m pivots smoothly about its midpoint. A child of mass 20kg sits on one end, and a child of mass $m$ sits 30cm away from the opposite end. The seesaw remains balanced. Find $m$.
	\item A light, horizontal shelf of length 5m is fastened at the left and has masses $m$, $2m$, and $3m$ suspended from it, at distances of 1m, 3m, and 5m respectively.
		\begin{enumerate}
			\item Find the moment about the fastening point.
			\item The fastening consists of a 10cm nail extending from the shelf into the wall, shown in red below. Modelling the reaction force of the wall on this nail as acting at a single point, 5cm into the wall, find the force exerted by the wall on this nail to keep the shelf stationary.
		\end{enumerate}
		\begin{center}
		\begin{tikzpicture}
			\draw (0,0) -- (1,0) -- (1,4) -- (0,4) -- (0,0);
			\node[left] at (0,2) {Wall};
			\draw[ultra thick] (1,2) -- (6,2);
			\node[right] at (6,2) {Shelf};
			\draw[red,thick] (1,2) -- (0.5,2);
			\node[red,below] at (0.5,2) {Nail};
			
			\draw[fill] (2,2) circle[radius=0.1];
			\draw[fill] (4,2) circle[radius=0.1];
			\draw[fill] (6,2) circle[radius=0.1];
		\end{tikzpicture}
		\end{center}
	\item A ramp of length 6m and mass 10kg rests with one end on rough ground and the other against a smooth wall. The angle formed by the ramp with the ground is $30^\circ$. A person of mass 70kg stands 5m up the ramp and the ramp is on the point of slipping against the ground. By modelling the ramp as a uniform rod and the person as a particle, find the coefficient of friction between the base of the ramp and the ground.
\end{enumerate}






\end{document}