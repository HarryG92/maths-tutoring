
\documentclass{article}

\usepackage[left=2cm,right=2cm, top=2cm, bottom = 2cm]{geometry}
\usepackage{amsfonts}

\usepackage{amsmath}
\usepackage{xcolor}

\usepackage{tikz}
\usepackage{subfigure}



\pagestyle{empty}

\setlength{\tabcolsep}{15pt}


\newcommand{\deriv}[3][]{\frac{\mathrm{d}^{#1}#2}{\mathrm{d}#3^{#1}}}
\newcommand{\diff}{\;\mathrm{d}}

\newcommand{\norm}[1]{\left|\kern-1pt\left|#1\right|\kern-1pt\right|}




\begin{document}

\title{Exponential Fourier Series; Fourier and Laplace Transforms}
\date{}

\maketitle
\thispagestyle{empty}

\Large

\textbf{\underline{Objective: To understand the Laplace transform and its relation}}

\textbf{\underline{to Fourier series.}}






\vspace{5mm}

















\textbf{Theory: Exponential Fourier Series:}

\bigskip


Recall the identities
\[\cos(t)=\frac{e^{jt}+e^{-jt}}{2},\qquad \sin(t)=\frac{e^{jt}-e^{-jt}}{2j}.\]

Using these, we can rewrite a Fourier series as a sum of exponentials; however, to do this, we must allow negative values of $n$, so our sum goes from $-\infty$ to $\infty$. Given a Fourier series
\[f(t)=\frac{a_0}{2}+\sum_{n=1}^\infty \left[a_n\cos\left(\frac{2\pi nt}{L}\right) + b_n\sin\left(\frac{2\pi nt}{L}\right)\right],\]
we substitute our above identities, and set $b_0=0$:
\begin{align*}
	f(t) &= \frac{a_0}{2} - \frac{b_0j}{2} + \sum_{n=1}^\infty \left[a_n\frac{e^{2\pi njt/L}+e^{-2\pi njt/L}}{2} + b_n\frac{e^{2\pi njt/L}-e^{-2\pi njt/L}}{2j}\right]\\
	&= \frac{a_0}{2} -\frac{b_0j}{2}+ \sum_{n=1}^\infty \left[\frac{a_n}{2}\left(e^{2\pi njt/L}+e^{-2\pi njt/L}\right) - \frac{b_nj}{2}\left(e^{2\pi njt/L}-e^{-2\pi njt/L}\right)\right]\\
	&= \frac{a_0-b_0j}{2} + \sum_{n=1}^\infty \frac{a_n-b_nj}{2}e^{2\pi njt/L} + \sum_{n=1}^\infty \frac{a_n+b_nj}{2}e^{-2\pi njt/L}
\end{align*}

Since we have negative powers of $e$ occuring, it makes sense to allow $n$ to take negative values; how do our Fourier coefficients behave with negative values of $n$? We see that, simply substituting $-n$ into the definition:
\begin{align*}
	a_{-n} &= \frac{2}{L}\int_a^{a+L} f(t)\cos\left(\frac{-2\pi nt}{L}\right)\diff t\\
	&=\frac{2}{L}\int_a^{a+L} f(t)\cos\left(\frac{2\pi nt}{L}\right)\diff t\\
	&=a_n,
\end{align*}
since cosine is an even function. For the sine coefficients, on the other hand, $b_{-n}=-b_n$, since sine is an odd function. So we can extend our definition of the Fourier coefficients to negative values of $n$, with the simple relations $a_{-n}=a_n$, and $b_{-n}=-b_n$.

Therefore, in the second infinite series of our rearranged Fourier series above, we can write
\begin{align*}
	\sum_{n=1}^\infty \frac{a_n+b_nj}{2}e^{-2\pi njt/L}&=\sum_{n=1}^\infty \frac{a_{-n}-b_{-n}j}{2}e^{-2\pi njt/L}\\
	&= \sum_{n=-\infty}^1 \frac{a_n-b_nj}{2}e^{2\pi njt/L}.
\end{align*}
This makes the summands equal to those of the first infinite series, but the limits now go from $-\infty$ to $1$, instead of $1$ to $\infty$. The constant term also has the same form, so we can treat it as
\[\frac{a_n-b_nj}{2}e^{2\pi njt/L}\]
when $n=0$. Therefore every term we are adding has the same form, so we can combine them all into a single infinite sum:
\[F(t)=\sum_{n=-\infty}^\infty \frac{a_n-b_nj}{2}e^{2\pi njt/L}.\]\medskip


Now consider the coefficients in the above series. From the definitions, we have
\begin{align*}
	\frac{a_n-b_nj}{2} &= \frac{1}{2}\left[\frac{2}{L}\int_a^{a+L}\!\!f(t)\cos\left(\frac{2\pi nt}{L}\right)\diff t -\frac{2j}{L}\int_a^{a+L}\!\! f(t)\sin\left(\frac{2\pi nt}{L}\right)\diff t\right]\\
	&=\frac{1}{L}\int_a^{a+L}\!\!f(t)\left[\cos\left(\frac{2\pi nt}{L}\right) - j \sin\left(\frac{2\pi nt}{L}\right)\right]\diff t\\
	&=\frac{1}{L}\int_a^{a+L}\!\! f(t)e^{-2\pi njt/L}\diff t.
\end{align*}

So we define
\[c_n=\frac{1}{L}\int_a^{a+L}\!\!f(t)e^{-2\pi njt/L}\diff t,\]
and write the \textbf{exponential Fourier series} of $f$ over $[a,a+L]$ as
\[f(t)=\sum_{n=-\infty}^\infty c_n e^{2\pi njt/L}.\]




\clearpage












\textbf{Theory: The Coefficient Function:}\bigskip


Consier the exponential Fourier coefficients, $c_n$. We can view this as a function which takes an integer $n$ as input and returns a complex number $c_n$ as output. Since this depends on the original function $f$, let us denote this coefficient function $\hat{f}$: $\hat{f}(n)=c_n$.

This process of ``putting a hat'' on a function $f$ is therefore a type of \textbf{operator}. You have seen operators before, such as differentiation; just as a function takes a number and gives you a number, an operator takes a function and gives you a new function. In this case, we start with a function $f$ and obtain a function $\hat{f}$. There is a crucial difference between this ``hat operator'' and the differentiation operator, however. Differentiation takes a function that has real inputs and real outputs, and it gives a function out that has real inputs and real outputs. The hat operator, on the other hand, takes a function with real inputs and outputs, and gives out a function with integer inputs and complex outputs!

The original function $f$ often has time as an input, in applications. The integer inputs to $\hat{f}$ are the harmonics over $L$, so represent frequencies. We say that $f$ is a function on the \textbf{time domain}, whereas $\hat{f}$ is a function on the \textbf{frequency domain}. In general, any real number is possible as the frequency of a sinusoid (or a complex exponential!), but $\hat{f}$ takes only integer multiples of the fundamental frequency on $L$. We say $\hat{f}$ is a function on the \textbf{discrete frequency domain}.

We shall see a generalisation of Fourier series later, the Fourier transform, which allows any real frequency as input, so is a function on the entire frequency domain, not just the integer multiples of a fixed fundamental frequency. The Fourier transform is a complex-valued function on the \textit{continuous} frequency domain.

It is worth looking more closely at the definition of our hat operator. It takes a function $f(t)$ and returns the function $\hat{f}(n)$, where
\[\hat{f}(n)=\frac{1}{L}\int_a^{a+L}\!\! f(t)e^{-2\pi njt/L}\diff t.\]

So we introduce a new variable $n$, then integrate with respect to the old variable to leave an expression using only the new variable. We multiply $f(t)$ by a function $e^{-2\pi njt/L}$ which involves \textit{both $t$ and $n$}, then integrate in the time domain to leave a function in the frequency domain. So the function $e^{-2\pi njt/L}$ acts as a bridge between the time and frequency domains.

To recover the original function, via its exponential Fourier series, we multiply by a similar ``bridge'' function, $e^{2\pi njt/L}$, and sum over all value of $n$---we sum over the frequency domain. This leaves us with $f_\mathrm{exp}(t)$, a function in the time domain, which (for sufficiently nice $f$), will be equal to the original function $f$.


\clearpage



\textbf{Theory: The Fourier Transform:}\bigskip


Fourier series are useful for studying periodic functions or functions on a bounded interval. So if we have a function $f:\mathbb{R}\to \mathbb{C}$ (or just to $\mathbb{R}$, it doesn't really matter) and we want to study with Fourier series, we can't study the whole thing, but we can take some large value of $L$ and look at $f$ between $-\frac{L}{2}$ and $\frac{L}{2}$, then take the Fourier series. So we can study $f$ via Fourier series over as large an interval as we like, so what if we let $L$ tend to infinity?

We have, for $-\frac{L}{2}\leq t\leq \frac{L}{2}$:
\begin{align*}
	f(t)&=\sum_{n=-\infty}^\infty c_{L,n} e^{2\pi i nt/L}\\
	&=\sum_{n=-\infty}^\infty \left(\frac{1}{L}\int_{-\frac{L}{2}}^{\frac{L}{2}}f(t)e^{-2\pi i nt/L}\diff t\right) e^{2\pi int/L}\\
	&= \sum_{n=-\infty}^\infty \left(\int_{-\frac{L}{2}}^{\frac{L}{2}} f(t) e^{-it\omega_n}\diff t \right) e^{it\omega_n}\frac{\omega_0}{2\pi},
\end{align*}
where $\omega_0=\frac{2\pi}{L}$ is the fundamental frequency and $\omega_n=n\omega_0$.

Now we can view the integral in the above expression as a function on the discrete frequency domain; given $\omega_n$ a frequency, we can take
\[\hat{f}(\omega_n)=\frac{1}{2\pi}\int_{-\frac{L}{2}}^{\frac{L}{2}} f(t)e^{-it\omega_n}\diff t.\]
Then the difference in frequency between two adjacent frequencies is $\omega_{n+1}-\omega_n = \omega_0$, so let us write the fundamental frequency $\omega_0$ as $\delta\omega$, the small change in frequency. Then the sum above samples the frequency function $\hat{f}$ at points $\omega_n$ a small distance $\delta\omega$ apart, multiplies the width $\delta\omega$ of that interval by the height of the frequency function $\hat{f}(\omega_n)$, and adds them all up. This sounds suspiciously like the definition of an integral! Indeed, as we let $L$ tend to infinity, $\delta\omega=\omega_0=\frac{2\pi}{L}$ tends to 0, so we are multiplying an infinitesimal width by the height of the function and summing:
\begin{align*}
	f(t)&=\lim_{L\to \infty}\sum_{n=-\infty}^\infty \left(\frac{1}{2\pi}\int_{-\frac{L}{2}}^{\frac{L}{2}} f(t) e^{-it\omega_n}\diff t \right) e^{it\omega_n}\delta\omega\\
	&=\int_{-\infty}^\infty \left(\frac{1}{2\pi}\int_{-\infty}^\infty f(t)e^{-it\omega}\diff t\right) e^{it\omega}\diff \omega.
\end{align*}


Therefore we define the \textbf{Fourier transform} of $f$ to be the function on the continuous frequency domain
\[\hat{f}(\omega)=\frac{1}{2\pi}\int_{-\infty}^\infty f(t)e^{-i\omega t}\diff t,\]
and can recover the original function $f$ by the \textbf{inverse Fourier transform}
\[f(t)=\int_{-\infty}^\infty \hat{f}(\omega)e^{i\omega t}\diff \omega.\]

Note that we take a function of $t$, introduce a new variable $\omega$, and integrate with respect to $t$, removing the old variable and leaving a function $\hat{f}$ purely of the new variable $\omega$; for the inverse transform, we use the same trick, introducing a new variable $t$ and integrating out $\omega$ to leave a function purely of $t$. The forward transform should be thought of as analagous to the coefficient function $c_n$ for Fourier series; indeed, it is almost exactly the same expression. The inverse transform should be thought of as analagous to the Fourier series itself, summing up each sinusoid with the amplitude given by the coefficient function; however, since we now allow any frequency $\omega\in \mathbb{R}$, instead of just integer multiples of the fundamental frequency, we have an integral instead of a sum.\bigskip



\textbf{Theory: The Laplace Transform:}\bigskip

The Fourier transform is a great idea in principle, but in practice it often isn't well-defined. Multiplying by an exponential and then integrating from $-\infty$ to $\infty$ doesn't usually give a finite answer! The Laplace transform is a variant of the Fourier transform that greatly enlarges the set of functions on which the transform can be applied.

There are two changes made to go from the Fourier transform to the Laplace: first, we only integrate from 0 to $\infty$ instead of $-\infty$ to $\infty$; secondly we replace the real frequency $\omega$ by a complex number $s=\rho + i \omega$. The effect is as follows:
\begin{align*}
	\mathcal{L}(f)&=\int_0^\infty f(t)e^{-st}\diff t\\
	&= \int_0^\infty f(t)e^{-\rho t}e^{-i\omega t}\diff t.
\end{align*}
The idea is that for large enough values of $\rho$ (the real part of $s$), the exponential decay term $e^{-\rho t}$ will be force the integrand to decay rapidly as $t\to \infty$ and allow the integral to converge.

So the Laplace transform should be thought of as a continuous analogue of the Fourier series, with some slight changes to get better convergence.








\end{document}