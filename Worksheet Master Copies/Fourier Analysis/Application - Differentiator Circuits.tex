
\documentclass{article}

\usepackage[left=2cm,right=2cm, top=2cm, bottom = 2cm]{geometry}
\usepackage{amsfonts}

\usepackage{amsmath}
\usepackage{xcolor}

\usepackage{tikz}
\usepackage{subfigure}



\pagestyle{empty}

\setlength{\tabcolsep}{15pt}


\newcommand{\deriv}[3][]{\frac{\mathrm{d}^{#1}#2}{\mathrm{d}#3^{#1}}}
\newcommand{\diff}{\;\mathrm{d}}

\newcommand{\norm}[1]{\left|\kern-1pt\left|#1\right|\kern-1pt\right|}
\newcommand{\bra}[1]{\left\langle #1 \,\right|}
\newcommand{\ket}[1]{\left|\, #1\right\rangle}
\newcommand{\braket}[2]{\left\langle #1 \mid #2 \right\rangle}




\begin{document}

\title{Application: Differentiator Circuits}
\date{}

\maketitle
\thispagestyle{empty}

\Large

\vskip -10mm

\textbf{\underline{Objective: To understand how the properties of the Fourier transform }}

\textbf{\underline{allow us to design a circuit that differentiates the input signal with}}

\textbf{\underline{respect to time.}}






\vspace{5mm}








Recall the Fourier transform (ordinary frequency version):
\[\hat{f}(\xi) = \int_{-\infty}^\infty f(t)e^{-2\pi j \xi t}\diff t,\]
and its inverse transform:
\[f(t)=\int_{-\infty}^\infty \hat{f}(\xi)e^{2\pi j\xi t}\diff \xi.\]
Recall also that
\[\widehat{\deriv{f}{t}}(\xi)=(2\pi j \xi)\hat{f}(\omega).\]\bigskip








The output voltage $V_\mathrm{out}$ of an op-amp (shown below) depends on the two input voltages $V_1$ and $V_2$ according to
\[V_\mathrm{out}=G(V_1-V_2),\]
where $G$ is some very large number (the gain of the amplifier).

\begin{center}
\begin{tikzpicture}[scale=0.8]
	\draw (0,0) -- (0,6) -- (5,3) -- (0,0);
	\draw (-1,2) -- (0,2);
	\node[left] at (-1,2) {$V_1$};
	\node[right] at (0,2) {$+$};
	\draw (-1,4) -- (0,4);
	\node[left] at (-1,4) {$V_2$};
	\node[right] at (0,4) {$-$};
	\draw (5,3) -- (6,3);
	\node[right] at (6,3) {$V_\mathrm{out}$};
\end{tikzpicture}
\end{center}

\clearpage


Suppose we connect the output of the op-amp through a voltage divider to the negative input, as shown, and tie the positive input to ground:

\begin{center}
\begin{tikzpicture}[scale=0.8]
	\draw (0,0) -- (0,6) -- (5,3) -- (0,0);
	\draw (-0.5,2) -- (0,2);
	\node[right] at (0,2) {$+$};
	\draw (-3,4) -- (-2,4) -- (-2,4.1) -- (-1,4.1) -- (-1,3.9) -- (-2,3.9) -- (-2,4.1) -- (-1,4.1) -- (-1,4) -- (0,4);
	\node[below] at (-1.5,4) {$R_1$};
	\node[left] at (-3,4) {$V_\mathrm{in}$};
	\node[right] at (0,4) {$-$};
	\draw (5,3) -- (6,3);
	\node[right] at (6,3) {$V_\mathrm{out}$};
	
	\draw (5.5,3) -- (5.5,7) -- (3,7) -- (3,6.9) -- (2,6.9) -- (2,7.1) -- (3,7.1) -- (3,6.9) -- (2,6.9) -- (2,7) -- (-0.5,7) -- (-0.5,4);
	\node[below] at (2.5,7) {$R_2$};
	
	\draw (-0.5,2) -- (-0.5,-1);
	\draw (-0.7,-1) -- (-0.3,-1);
	\draw (-0.6,-1.1) -- (-0.4,-1.1);
	\draw (-0.55,-1.2) -- (-0.45,-1.2);
\end{tikzpicture}
\end{center}

We assume that the input impedance of the op-amp is extremely high, so no current flows into or out of the op-amp inputs. Let $V_0$ denote the voltage at the negative input of the op-amp.

\begin{enumerate}
	\item By considering current, show that
		\[\frac{V_\mathrm{out}-V_0}{R_2}=\frac{V_0-V_\mathrm{in}}{R_1}.\]
	\item Hence show that
		\[V_\mathrm{out} = V_0 \left(1+\frac{R_2}{R_1}\right) - \frac{R_2}{R_1}V_\mathrm{in}.\]
	\item By considering the effect on $V_\mathrm{out}$ if $V_0>0$ and if $V_0<0$ respectively, show that the action of the circuit is to make $V_0=0$.
	\item Hence conclude that the output of the circuit is given by
		\[V_\mathrm{out}=-\frac{R_2}{R_1}V_\mathrm{in}.\]
	\item Now suppose that $R_1$ is replaced by a capacitor, $C$. Recall that the impedance of a capacitor when a sinusoidal voltage of frequency $\xi$ is applied is given by
		\[\frac{1}{2\pi \xi j C}.\]
		\begin{center}
		\begin{tikzpicture}[scale=0.7]
			\draw (0,0) -- (0,6) -- (5,3) -- (0,0);
			\draw (-0.5,2) -- (0,2);
			\node[right] at (0,2) {$+$};
			\draw (-3,4) -- (-2,4) -- (-2,4.5) -- (-2,3.5);
			\draw (-1.8,3.5) -- (-1.8,4.5) -- (-1.8,4) -- (-1,4) -- (0,4);
			\node[below] at (-1.5,4) {$C$};
			\node[left] at (-3,4) {$V_\mathrm{in}$};
			\node[right] at (0,4) {$-$};
			\draw (5,3) -- (6,3);
			\node[right] at (6,3) {$V_\mathrm{out}$};
			
			\draw (5.5,3) -- (5.5,7) -- (3,7) -- (3,6.9) -- (2,6.9) -- (2,7.1) -- (3,7.1) -- (3,6.9) -- (2,6.9) -- (2,7) -- (-0.5,7) -- (-0.5,4);
			\node[below] at (2.5,7) {$R$};
			
			\draw (-0.5,2) -- (-0.5,-1);
			\draw (-0.7,-1) -- (-0.3,-1);
			\draw (-0.6,-1.1) -- (-0.4,-1.1);
			\draw (-0.55,-1.2) -- (-0.45,-1.2);
		\end{tikzpicture}
		\end{center}
		Show that if $V_\mathrm{in}$ is a sinusoid of amplitude $A$, then the amplitude of $V_\mathrm{out}$ is given by
		\[-2\pi \xi jRC A.\]
	\item Suppose that $V_\mathrm{in}=f(t)$ is some time-varying function. Show that the Fourier transform of $V_\mathrm{out}$ is given by
		\[-2\pi\xi j RC\hat{f}(\omega).\]
	\item Hence conclude that the (time-domain) output of the circuit is
		\[V_\mathrm{out}(t)=-RC\deriv{V_\mathrm{in}}{t}.\]
	\item Suppose instead that we leave $R_1$ as a resistor and replace instead $R_2$ by the capacitor $C$, as shown below. Suppose the circuit is switched on at time 0, and before that has 0V input and output. By carrying out a similar frequency-domain analysis, show that the output of the circuit is now given by
		\[V_\mathrm{out}(t)=-\frac{1}{RC}\int_0^t V_\mathrm{in}(\tau)\diff \tau,\]
		where $\tau$ is a dummy variable of integration.
		\begin{center}
		\begin{tikzpicture}[scale=0.7]
			\draw (0,0) -- (0,6) -- (5,3) -- (0,0);
			\draw (-0.5,2) -- (0,2);
			\node[right] at (0,2) {$+$};
			\draw (-3,4) -- (-2,4) -- (-2,4.1) -- (-1,4.1) -- (-1,3.9) -- (-2,3.9) -- (-2,4.1) -- (-1,4.1) -- (-1,4) -- (0,4);
			\node[below] at (-1.5,4) {$R$};
			\node[left] at (-3,4) {$V_\mathrm{in}$};
			\node[right] at (0,4) {$-$};
			\draw (5,3) -- (6,3);
			\node[right] at (6,3) {$V_\mathrm{out}$};
	
			\draw (5.5,3) -- (5.5,7) -- (3,7) -- (3,7.5) -- (3,6.5);
			\draw (2.8,6.5) -- (2.8,7.5) -- (2.8,7) -- (-0.5,7) -- (-0.5,4);
			\node[below] at (2.5,7) {$C$};
	
			\draw (-0.5,2) -- (-0.5,-1);
			\draw (-0.7,-1) -- (-0.3,-1);
			\draw (-0.6,-1.1) -- (-0.4,-1.1);
			\draw (-0.55,-1.2) -- (-0.45,-1.2);
		\end{tikzpicture}
		\end{center}
\end{enumerate}








\end{document}