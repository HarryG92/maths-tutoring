
\documentclass{article}

\usepackage[left=2cm,right=2cm, top=2cm, bottom = 2cm]{geometry}
\usepackage{amsfonts}

\usepackage{amsmath}
\usepackage{amssymb}
\usepackage{xcolor}

\usepackage{tikz}
\usepackage{subfigure}



\pagestyle{empty}

\setlength{\tabcolsep}{15pt}


\newcommand{\deriv}[3][]{\frac{\mathrm{d}^{#1}#2}{\mathrm{d}#3^{#1}}}
\newcommand{\diff}{\;\mathrm{d}}

\newcommand{\norm}[1]{\left|\kern-1pt\left|#1\right|\kern-1pt\right|}
\newcommand{\bra}[1]{\left\langle #1 \,\right|}
\newcommand{\ket}[1]{\left|\, #1\right\rangle}
\newcommand{\braket}[2]{\left\langle #1 \mid #2 \right\rangle}

\let\take\setminus
\let\lamdba\lambda


\begin{document}

\title{Lebesgue Integration Solutions}
\date{}

\maketitle
\thispagestyle{empty}

\Large



The solutions provided are sketch solutions only, and are not entirely rigourous.



\begin{enumerate}
	\item In this exercise, we prove that the Lebesgue-measurable sets form a $\sigma$-algebra and that the Lebesgue measure is indeed a measure.
		\begin{enumerate}
			\item First we establish some properties of the Lebesgue outer measure. Let $\lambda^+$ be the Lebesgue outer measure:
				\[\lambda^+(E)=\inf\left\{\sum_{n=1}^\infty (b_n-a_n):E\subseteq \bigcup_{n=1}^\infty (a_n,b_n)\right\}.\]
				Show that:
				\begin{enumerate}
					\item $\lambda^+(\varnothing)=0$;
						
						{\color{blue}
							We can cover the empty set with a single interval $\left(0,\frac{1}{k}\right)$, say, for any $k$. So $\lambda^+(\varnothing)\leq \frac{1}{k}$ for all $k$, so $\lambda^+(\varnothing)=0$.
						}
					
					\item If $A\subseteq B\subseteq \mathbb{R}$, then $\lambda^+(A)\leq\lambda^+(B)$;
					
						{\color{blue}
							Suppose for a contradiction that $\lambda^+(A)> \lambda^+(B)$. Then, since $\lambda^+(B)$ is the infimum (\textit{greatest} lower bound) on the sums of lengths of sequences of intervals covering $B$, there must be a sequence of intervals $(a_n,b_n)$ covering $B$ with
							\[\sum_{n=1}^\infty (b_n-a_n) < \lambda^+(A).\]
							But $A\subseteq B$, so the same sequence of intervals must cover $A$, meaning that, by definition,
							\[\lambda^+(A)\leq\sum_{n=1}^\infty (b_n-a_n),\]
							a contradiction.
						}
					
					\item If $A_1,A_2,\hdots$ are subsets of $\mathbb{R}$, then
						\[\lambda^+\left(\bigcup_{n=1}^\infty A_n\right) \leq \sum_{n=1}^\infty \lambda^+ (A_n).\]
						
						{\color{blue}
							For any collection of sequences of intervals $I_{n,k}$ such that
							\[A_n\subseteq \bigcup_{k=1}^\infty I_{n,k}\]
							for each $n$, we have
							\[\bigcup_{n=1}^\infty \subseteq \bigcup_{n=1}^\infty\bigcup_{k=1}^\infty I_{n,k}.\]
							Since a countable union of countable sets is countable, this is still a countable collection of intervals (all $I_{n,k}$ as $n$ and $k$ range over positive integers), so
							\[\lambda^+\left(\bigcup_{n=1}^\infty\right) \leq \sum_{n=1}^\infty \sum_{k=1}^\infty (b_{n,k}-a_{n,k}),\]
							where $I_{n,k}=(a_{n,k},b_{n,k})$. For each $n$, $\lambda^+(A_n)$ is the greatest lower bound on expressions of the form
							\[\sum_{k=1}^\infty (b_{n,k}-a_{n,k}),\]
							so the sum of the $\lambda^+(A_n)$'s is the greatest lower bound on sums of such expressions. Therefore
							\[\lambda^+\left(\bigcup_{n=1}^\infty\right)\leq \sum_{n=1}^\infty \lambda^+(A_n).\]
						}
						
				\end{enumerate}
			\item Next we show that the Lebesgue-measurable sets form a $\sigma$-algebra. Recall that a set $E$ is Lebesgue-measurable if for any other set $A\subseteq \mathbb{R}$,
				\[\lambda^+(A)=\lambda^+\left(A\cap E\right) + \lambda^+\left(A\cap (\mathbb{R}\take E)\right).\]
				Show that:
				\begin{enumerate}
					\item $\mathbb{R}$ is Lebesgue-measurable.
					
						{\color{blue}
							Let $A$ be any subset of $\mathbb{R}$. Then $A\cap \mathbb{R}=A$ and $A\cap (\mathbb{R}\take\mathbb{R})=\varnothing$. So
							\begin{align*}
								\lambda^+(A\cap \mathbb{R})+\lambda^+(A\cap(\mathbb{R}\take\mathbb{R}))&=\lambda^+(A)+\lambda^+(\varnothing)\\
								&=\lambda^+(A).
							\end{align*}
						}
					
					\item If $E$ is Lebesgue-measurable, then so is $\mathbb{R}\take E$.
					
						{\color{blue}
							Suppose $E$ is Lebesgue-measurable and let $F=\mathbb{R}\take E$. Then for any $A\subseteq \mathbb{R}$, we have
							\begin{align*}
								\lambda^+(A)&=\lambda^+(A\cap E)+\lambda^+(A\cap(\mathbb{R}\take E))\\
								&=\lambda^+(\mathbb{R}\take F)+\lambda^+(F),
							\end{align*}
							so $F$ is Lebesgue-measurable.
						}
					
					\item If $E_1,E_2,\hdots$ are Lebesgue-measurable, then so is
						\[\bigcup_{n=1}^\infty E_n.\]
					
						{{\color{red}When I wrote this question, I knew it was hard. When I actually tried to do it myself, I realised it's EXTREMELY hard! I had to get a hint from Google to help me through. Sorry for setting such a hard question!\bigskip}\color{blue}
						
						We proceed in several steps. First we work with only finitely many $E_n$'s, and assume they are pairwise disjoint. Then we extend to infinitely many pairwise disjoint $E_n$'s. Finally, we show that given any $E_n$'s, we can replace them by pairwise disjoint ones, so our proof in that case applies.\medskip
						
						
							First we show that if $E_1$ and $E_2$ are disjoint and Lebesgue-measurable, then so is $E_1\cup E_2$. Since $E_1$ is Lebesgue-measurable, we have, for any $A\subseteq \mathbb{R}$
							\[\lambda^+(A)=\lambda^+(A\cap E_1)+\lambda^+(A\cap (\mathbb{R}\take E_1)).\]
							Now we use measurability of $E_2$ to see that
							\[\lambda^+(A\cap B)= \lambda^+((A\cap B)\cap E_2)+\lambda^+((A\cap B)\cap(\mathbb{R}\take E_2))\]
							both for $B=E_1$ and for $B=\mathbb{R}\take E_1$. Therefore
							\begin{align*}
								\lambda^+(A)&= \lambda^+((A\cap E_1)\cap E_2)+\lambda^+((A\cap E_1)\cap(\mathbb{R}\take E_2))\\
								&{\color{white} =\quad} +\lambda^+((A\cap (\mathbb{R}\take E_1))\cap E_2)+\lambda^+((A\cap (\mathbb{R}\take E_1))\cap(\mathbb{R}\take E_2)).
							\end{align*}
							Since $E_1$ and $E_2$ are disjoint, $E_1\cap E_2=\varnothing$, $E_1\cap (\mathbb{R}\take E_2)=E_1$, and $(\mathbb{R}\take E_1)\cap E_2=E_2$. Also, by de Morgan's Laws, $(\mathbb{R}\take E_1)\cap (\mathbb{R}\take E_2)=\mathbb{R}\take (E_1\cup E_2)$. Therefore we can rewrite the above as
							\[\lambda^+(A)=\lambda^+(A\cap\varnothing)+\lambda^+(A\cap E_1)+\lambda^+(A\cap E_2)+\lambda^+(A\cap\mathbb{R}\take(E_1\cup E_2)).\]
							Now, $A\cap(E_1\cup E_2)=(A\cap E_1)\cup(A\cap E_2)$, so by part (a,iii), we have $\lambda^+(A\cap(E_1\cup E_2))\leq \lambda^+(A\cap E_1)+\lambda^+(A\cap E_2)$. Substituting this in the above, we have
							\[\lambda^+(A)\geq \lambda^+(A\cap(E_1\cup E_2))+\lambda^+(A\cap\mathbb{R}\take(E_1\cup E_2)).\]
							But, at the same time, $A=(A\cap(E_1\cup E_2))\cup(A\cap\mathbb{R}\take(E_1\cup E_2))$, so by part (a,iii) again,
							\[\lambda^+(A)\leq  \lambda^+(A\cap(E_1\cup E_2))+\lambda^+(A\cap\mathbb{R}\take(E_1\cup E_2)).\]
							Putting these two inequalities together,
							\[\lambda^+(A)= \lambda^+(A\cap(E_1\cup E_2))+\lambda^+(A\cap\mathbb{R}\take(E_1\cup E_2)).\]
							
							\bigskip
							
							So, after rather a lot of work, we have shown that if $E_1$ and $E_2$ are disjoint and Lebesgue-measurable, then $E_1\cup E_2$ is Lebesgue-measurable. Moreover, notice from the proof that we obtain two expressions for $\lambda^+(A)$; setting these equal to each other and cancelling $\lambda^+(A\cap\mathbb{R}\take(E_1\cup E_2))$, we conclude that
							\[\lambda^+(A\cap(E_1\cup E_2))=\lambda^+(A\cap E_1)+\lambda^+(A\cap E_2).\]
							
							A simple induction argument now proves that if $E_1,E_2,\hdots,E_n$ are pairwise disjoint and Lebesgue-measurable, then their union is Lebesgue-measurable too, and for any $A\subseteq \mathbb{R}$,
							\begin{equation*}
								\lambda^+\left(A\cap\bigcup_{k=1}^n E_k\right)=\sum_{k=1}^n \lambda^+(A\cap E_k).\tag{$\star$}
							\end{equation*}
							
							Now we extend to countably infinite collections of pairwise disjoint, measurable sets. So suppose that $E_1,E_2,\hdots$ are pairwise disjoint and measurable. Then for any set $A$, we have
							\[A\cap\bigcup_{n=1}^\infty E_n=\bigcup_{n=1}^\infty (A\cap E_n),\]
							so by (a,iii),
							\begin{align*}
								\lambda^+\left(A\cap\bigcup_{n=1}^\infty E_n\right)&\leq \sum_{n=1}^\infty \lambda^+ (A\cap E_n)\\
								&=\lim_{N\to\infty}\sum_{n=1}^N \lambda^+(A\cap E_n)\\
								&=\lim_{N\to \infty}\lambda^+\left(A\cap\bigcup_{n=1}^N E_n\right),
							\end{align*}
							where we used equation ($\star$) in the last line. Now,
							\[A\cap\bigcup_{n=1}^N E_n\subseteq A\cap \bigcup_{n=1}^\infty E_n,\]
							so by part (a,ii), we have
							\[\lambda^+\left(A\cap\bigcup_{n=1}^N E_n\right)\leq \lambda^+\left(A\cap\bigcup_{n=1}^\infty E_n\right).\]
							Therefore, taking the limit as $N\to \infty$, we have
							\[\lim_{N\to \infty}\lambda^+\left(A\cap\bigcup_{n=1}^N E_n\right)\leq \lambda^+\left(A\cap\bigcup_{n=1}^\infty E_n\right).\]
							But we have already shown that the reverse of this inequality holds, and that the left-hand side is equal to
							\[\sum_{n=1}^\infty \lambda^+(A\cap E_n),\]
							so we must have
							\begin{equation*}
								\sum_{n=1}^\infty \lambda^+(A\cap E_n)= \lambda^+\left(A\cap\bigcup_{n=1}^\infty E_n\right).\tag{$\star\star$}
							\end{equation*}
							
							\bigskip
							
							Now, we have already shown that the union of finitely many of the $E_n$ is measurable, so
							\begin{align*}
								\lambda^+(A)&=\lambda^+\left(A\cap\bigcup_{n=1}^N E_n\right)+\lambda^+\left(A\cap\left(\mathbb{R}\take\bigcup_{n=1}^N E_n\right)\right)\\
								&\geq\lambda^+\left(A\cap\bigcup_{n=1}^N E_n\right)+\lambda^+\left(A\cap\left(\mathbb{R}\take\bigcup_{n=1}^\infty E_n\right)\right) ,
							\end{align*}
							using part (a,ii) with the inclusion
							\[A\cap\left(\mathbb{R}\take\bigcup_{n=1}^\infty E_n\right)\subseteq A\cap\left(\mathbb{R}\take\bigcup_{n=1}^N E_n\right).\]
							Therefore, using equation ($\star$), we have
							\[\lambda^+(A)\geq \sum_{n=1}^N \lambda^+(A\cap E_n)+\lambda^+\left(A\cap\left(\mathbb{R}\take\bigcup_{n=1}^\infty E_n\right)\right).\]
							Taking the limit as $N\to \infty$, we have
							\begin{align*}
								\lambda^+(A)&\geq \sum_{n=1}^\infty \lambda^+(A\cap E_n)+\lambda^+\left(A\cap\left(\mathbb{R}\take\bigcup_{n=1}^\infty E_n\right)\right)\\
								&=\lambda^+\left(A\cap\bigcup_{n=1}^\infty E_n\right)+\lambda^+\left(A\cap\left(\mathbb{R}\take\bigcup_{n=1}^\infty E_n\right)\right),
							\end{align*}
							by equation ($\star\star$). The reverse of this inequality is true by a simple application of part (a,ii), so (finally!) we have proved that for $E_1,E_2,\hdots$ pairwise disjoint and Lebesgue-measurable, the union is also Lebesgue-measurable.\medskip
							
							Finally, it remains to deal with the case where the $E_1,E_2,\hdots$ are \textit{not} pairwise disjoint. Thankfully, this is now easy. Define
							\[F_n=E_n\take\bigcup_{k=1}^n E_k,\]
							so $F_n$ is the part of $E_n$ that isn't already in the preceding $E_k$'s. Then $F_1,F_2,\hdots$ are pairwise disjoint by construction, and
							\[\bigcup_{n=1}^\infty F_n=\bigcup_{n=1}^\infty E_n.\]
							Moreover, each $F_n$ is Lebesgue-measurable, by using parts (b,i) and (b,ii). So we can just replace all our $E_n$'s by $F_n$'s and then use our result for pairwise disjoint, measurable sets.
						}
					
				\end{enumerate}
			\item Finally we show that, when restricted to the $\sigma$-algebra of Lebesgue-meas\-ur\-a\-ble sets, the Lebesgue outer measure is actually a measure. We already know that $\lambda^+(\varnothing)=\varnothing$, so we only need to check that if $E_1,E_2,\hdots$ are Lebesgue-measurable \textit{and pairwise disjoint}, then
				\[\lambda^+\left(\bigcup_{n=1}^\infty E_n\right) = \sum_{n=1}^\infty \lambda^+(E_n).\]
				
						{\color{blue}
							We've actually already shown this! It was equation ($\star\star$) in the proof of part (b,iii). So at least all that hard work there saved us some work here!
						}
					
		\end{enumerate}
	\item In this exercise, we establish some more properties of Lebesgue measure and measurable sets. Show that:
		\begin{enumerate}
			\item Any interval $I$ is Lebesgue-measurable.
			
						{\color{blue}
							We need to show that for any set $A$,
							\[\lambda^+(A)=\lambda^+(A\cap I)+\lambda^+(A\cap(\mathbb{R}\take I)).\]
							First note that, since $I$ is an interval, $\mathbb{R}\take I=I_1\cup I_2$, where $I_1$ and $I_2$ are both intervals (or empty). This is shown just by checking all the possible cases (since $I$ could be $(-\infty,a)$, $(-\infty,a]$, $(a,b)$, $[a,b)$, etc.), but is easy in each case; for instance, if $I=(a,b)$, then $\mathbb{R}\take I=(-\infty,a]\cup[b,\infty$.
							
							Now take a cover of $A$ by a sequence of intervals $J_n$; so
							\[A\subseteq \bigcup_{n=1}^\infty J_n.\]
							Then
							\[A\cap I\subseteq \bigcup_{n=1}^\infty (J_n\cap I)\]
							and
							\[A\cap(\mathbb{R}\take I)\subseteq\bigcup_{n=1}^\infty (J_n\cap(I_1\cup I_2)).\]
							Moreover, the intersection of two intervals ($J_n$ and $I$, $I_1$, or $I_2$) is itself an interval; so we have covered $A\cap I$ and $A\cap(\mathbb{R}\take I)$ by sequences of intervals. Therefore, by definition of the outer measure as the infimum, and writing $l$ for the length of an interval,
							\begin{align*}
								\lambda^+(A\cap I)+\lambda^+(A\cap(\mathbb{R}\take I))&\leq \sum{n=1}^\infty l(J_n\cap I) + \sum_{n=1}^\infty [l(J_n\cap I_1)+l(J_n\cap I_2)]\\
								&=\sum_{n=1}^\infty[l(J_n\cap I)+l(J_n\cap I_1) + l(J_n\cap I_2)].
							\end{align*}
							But $I$, $I_1$, and $I_2$ are pairwise disjoint intervals, and $I\cup I_1\cup I_2=\mathbb{R}$, so $l(J_n\cap I)+l(J_n\cap I_1)+l(J_n\cap I_2)=l(J_n)$; so
							\[\lambda^+(A\cap I)+\lambda^+(A\cap(\mathbb{R}\take I))\leq \sum_{n=1}^\infty l(J_n).\]
							This is true for any sequence of intervals $J_n$ covering $A$, so taking the infimum
							\[\lambda^+(A\cap I)+\lambda^+(A\cap(\mathbb{R}\take I))\leq \lamdba^(A).\]
							The reverse inequality holds by question 1(a,ii), so we have
							\[\lambda^+(A\cap I)+\lambda^+(A\cap(\mathbb{R}\take I))= \lamdba^(A),\]
							so $I$ is measurable.
						}
					
			\item For any set $A\subseteq \mathbb{R}$, the outer measure can be found by approximating $A$ by arbitrary Lebesgue-measurable sets, not just intervals:
				\[\lambda^+(A)=\inf\left\{\sum_{n=1}^\infty \lambda(E_n):A\subseteq \bigcup_{n=1}^\infty E_n\right\},\]
				where the $E_n$ are taken to be Lebesgue-measurable.
				
						{{\color{red} HARD}
						
						\color{blue}
							The idea is that we show these are equal by showing the first is greater than or equal to the second, then show it is not strictly greater.
							
							By part (i), every interval is measurable, so a cover of $A$ by intervals is just a special case of a cover by measurable sets, so
							\[\left\{\sum_{n=1}^\infty \lambda(I_n):A\subseteq \bigcup_{n=1}^\infty I_n\right\}\subseteq \left\{\sum_{n=1}^\infty \lambda(E_n):A\subseteq \bigcup_{n=1}^\infty E_n\right\},\]
							where $I_n$ denotes intervals and $E_n$ any measurable sets. Since the second set has more elements, it has the possibility of containing smaller elements, but contains elements at least as small as any in the first set, so its infimum is less than or equal to the infimum of the first set:
							\[\lambda^+(A)\geq \inf\left\{\sum_{n=1}^\infty \lambda(E_n):A\subseteq \bigcup_{n=1}^\infty E_n\right\}.\]

							Now suppose for a contradiction that this inequality is strict. Then there is a sequence of measurable sets $E_n$ covering $A$ such that
							\[\lambda^+(A)>\sum_{n=1}^\infty \lambda(E_n).\]
							Let $\epsilon$ be the difference, so
							\[\epsilon=\lambda^+(A)-\sum_{n=1}^\infty \lambda(E_n).\]
							Since $\lambda(E_n)$ is the infimum of lengths of coverings of $E_n$ by intervals, for each $n$ we can take a sequence $I_{n,1},I_{n,2},\hdots$ of intervals covering $E_n$ such that
							\[\lambda(E_n)>\sum_{k=1}^\infty \lambda(I_{n,k})-2^{-n}\epsilon.\]
							The idea here is that essentially $\lambda(E_n)$ is the lower limit of coverings of $E_n$ by sequences of intervals, so we can get as close as we like to $\lambda(E_n)$ (within $2^{-n}\epsilon$) by taking a sufficiently good covering of $E_n$ by intervals. Then
							\begin{align*}
								\sum_{n=1}^\infty \lambda(E_n) &>\sum_{n=1}^\infty\left(\sum_{n=k}^\infty \lambda(I_{n,k})-2^{-n}\epsilon\right)\\
								&=\sum_{n=1}^\infty \sum_{k=1}^\infty \lambda(I_{n,k}) - \epsilon,
							\end{align*}
							using the sum of the geometric series $2^{-n}$.  Now, since the $I_{n,k}$ cover $E_n$, and the $E_n$ cover $A$, the $I_{n,k}$ (over all $n$ and $k$) cover $A$, so
							\begin{align*}
								\lambda^+(A)&\leq \sum_{n=1}^\infty\sum_{k=1}^\infty \lambda(I_{n,k})\\
								&< \sum_{n=1}^\infty \lambda(E_n)+\epsilon,
							\end{align*}
							but this contradicts the definition of $\epsilon$. So in fact the inequality from the start of the proof is not strict, so is an equality:
							\[\lambda^+(A)\geq \inf\left\{\sum_{n=1}^\infty \lambda(E_n):A\subseteq \bigcup_{n=1}^\infty E_n\right\}.\]
						}
					
		\end{enumerate}
	\item In this exercise we show that Lebesgue integration for non-negative, measurable functions is well-defined. Recall that for such a function the Lebesgue integral over an interval $I$ is defined to be
		\[\int_If(x)\diff x =\lim_{n\to \infty} \sum_{k=1}^\infty k2^{-n}m_{n,k}(f),\]
		where $m_{n,k}(f)=\lambda\left(f^{-1}[k2^{-n},(k+1)2^{-n}]\right)$ is the Lebesgue measure of the preimage of the interval $[k2^{-n},(k+1)2^{-n}]$ under $f$.
		\begin{enumerate}
			\item Show that
				\begin{align*}
					f^{-1}[k2^{-n},(k+1)2^{-n}]&=f^{-1}[(2k)2^{-(n+1)},(2k+1)2^{-(n+1)}]\\
					&{\color{white} = =}\cup f^{-1}[(2k+1)2^{-(n+1)},(2k+2)2^{-(n+1)}].
				\end{align*}
				
						{\color{blue}
							Let $A=[(2k)2^{-(n+1)},(2k+1)2^{-(n+1)}]$ and $B=[(2k+1)2^{-(n+1)},(2k+2)2^{-(n+1)}]$. Then we need to show that $f^{-1}(A\cup B)=f^{-1}(A)\cup f^{-1}(B)$. This is a general fact about preimages of sets and has nothing to do with the particular sets here. We want to show that two sets are equal, so we show that every element of $f^{-1}(A\cup B)$ is an element of $f^{-1}(A)\cup f^{-1}(B)$, and then show that every element of $f^{-1}(A)\cup f^{-1}(B)$ is an element of $f^{-1}(A\cup B)$.
							
							So let $x\in f^{-1}(A\cup B)$. This means that $f(x)\in A\cup B$, so $f(x)\in A$ or $f(x)\in B$ (or both). Therefore $x\in f^{-1}(A)$ or $x\in f^{-1}(B)$, so $x\in f^{-1}(A)\cup f^{-1}(B)$.
							
							Now let $x\in f^{-1}(A)\cup f^{-1}(B)$. We need to show that $x\in f^{-1}(A\cup B)$. This is very similar to the above, so I'll leave it for you.
						}
					
			\item Hence show that
				\[k2^{-n}m_{n,k}(f)\leq (2k)2^{-(n+1)}m_{n+1,2k}(f)+(2k+1)2^{-(n+1)}m_{n+1,2k+1}(f).\]
				
						{\color{blue}
							We can rewrite the right-hand side as
							\[k2^{-n}m_{n+1,2k}(f)+k2^{-n}m_{n+1,2k+1}(f)+2^{-(n+1)}m_{n+1,2k+1}(f).\]
							Now, by taking $\lambda$ of part (a), $m_{n,k}(f)\leq m_{n+1,2k}(f)+m_{n+1,2k+1}(f)$, so multiply this by $k2^{-n}$; then adding $2^{-(n+1)}m_{n+1,2k+1}(f)$ onto the right-hand side preserves the inequality, since $m_{n+1,2k+1}(f)\geq 0$.
						}
					
			\item Hence show that
				\[\sum_{k=1}^\infty k2^{-n}m_{n,k}(f)\leq \sum_{k=1}^\infty k2^{-(n+1)}m_{n+1,k}(f).\]
				
						{\color{blue}
							In the first sum, split off the first term and group together the remaining terms in pairs, so
							\begin{align*}
								\sum_{k=1}^\infty k2^{-(n+1)}m_{n+1,k}(f)&=2^{-(n+1)}m_{n+1,1}(f)\\
								&\quad+\sum_{k=1}^\infty \left[(2k)2^{-(n+1)}m_{n+1,2k}(f)\right.\\
								&{\color{white} \quad + \sum_{k=1}^\infty [ ]}\left.+(2k+1)2^{-(n+1)}m_{n+1,2k+1}(f)\right].
							\end{align*}
							Now applying part (b) for each $k$ gives the result.
						}
					
			\item Hence conclude that as $n\to\infty$, 
				\[\sum_{k=1}^\infty k2^{-n}m_{n,k}(f)\]
				converges either to a finite limit or to infinity, and hence the Lebesgue integral of a non-negative, measurable function is well-defined.
				
						{\color{blue}
							By part (c), as we increase $n$, the value of
							\[\sum_{k=1}^\infty k2^{-n}m_{n,k}(f)\]
							does not decrease. Therefore it is a non-decreasing sequence in $n$, and so either grows without bound (tends to infinity), or is bounded above and hence converges (if you don't know why this is, look up the Monotone Convergence Theorem).
						}
					
		\end{enumerate}
	\item In this question, we establish some properties of measurable functions and show that various functions $\mathbb{R}\to\mathbb{R}$ are measurable. Recall that $f$ is measurable if $f^{-1}[a,b]$ is measurable for any interval $[a,b]$. Show that:
		\begin{enumerate}
			\item If $f$ is measurable, then the preimage (under $f$) of any measurable set is measurable.

						{{\color{red} VERY HARD}
						
						\color{blue}
							The idea is that a measurable set is one that can be approximated well by intervals, and since $f$ is measurable, the preimage of an interval is measurable, so if we approximate a measurable set by intervals and take its preimage, we should get something measurable. Here's a sketch proof:\medskip
							
							Let $\Sigma=\{E:f^{-1}(E)\mbox{ is Lebesgue-measurable}\}$ be the set of all sets whose preimage is measurable. We show that $\Sigma$ is a $\sigma$-algebra. We need to check that $\mathbb{R}$ is in $\Sigma$ and that any countable union of $\Sigma$-sets is a $\Sigma$-set. For the first, note that $\mathbb{R}=f^{-1}(\mathbb{R})$, and since $\mathbb{R}$ is measurable, this proves $\mathbb{R}\in \Sigma$. Closure under countable unions follows from the fact that
							\[f^{-1}\left(\bigcup_{n=1}^\infty E_n\right)=\bigcup_{n=1}^\infty f^{-1}(E_n).\]
							
							So $\Sigma$ is a $\sigma$-algebra. Moreover, since $f$ is measurable, $f^{-1}(I)$ is measurable for any interval $I$, so any interval $I$ is in $\Sigma$. So the smallest $\sigma$-algebra containing all intervals (called the Borel $\sigma$-algebra) must be contained in $\Sigma$. So any Borel-measurable set $B$ is in $\Sigma$, but by definition of $\Sigma$, this means $f^{-1}(B)$ is measurable. Then it suffices to prove that once $\Sigma$ contains all Borel-measurable sets, it must contain all Lebesgue measurable sets. Details to follow...
							
							\iffalse%%%%%%%%%%%
							So let $E$ be a measurable set. To show that $f^{-1}(E)$ is measurable, we need to show that for any $A\subseteq \mathbb{R}$,
							\[\lambda^+(A)=\lamdba^+(A\cap f^{-1}(E))+\lambda^+(A\cap(\mathbb{R}\take f^{-1}(E))).\]
							
							Take a sequence of intervals $I_n$ which cover $E$, so
							\[E\subseteq \bigcup_{n=1}^\infty I_n.\]
							Then, taking preimages,
							\[f^{-1}(E)\subseteq \bigcup_{n=1}^\infty f^{-1}(I_n),\]
							because if $x\in f^{-1}(E)$, then $f(x)\in E$, so $f(x)$ is in the union of the $I_n$, so $f(x)$ is in some particular $I_k$, so $x\in f^{-1}(I_k)$, so $x$ is in the union of the $f^{-1}(I_n)$. Therefore
							\[A\cap f^{-1}(E)\subseteq A\cap\bigcup_{n=1}^\infty f^{-1}(I_n)=\bigcup_{n=1}^\infty (A\cap f^{-1}(I_n)).\]
							Taking outer measures and using question 1(a,iii):
							\[\lambda^+(A\cap f^{-1}(E))\leq \sum_{n=1}^\infty \lambda^+(A\cap f^{-1}(I_n)).\]
							Since $f$ is measurable and each $I_n$ is an interval, $f^{-1}(I_n)$ is measurable for each $n$. Moreover, $f^{-1}(E)$ is covered by the sets $f^{-1}(I_n)$, since $E$ is covered by the $I_n$. So we \fi%%%%%%%%%%%%
						}
					
			\item If $f$ is continuous then $f$ is measurable.
			
						{\color{blue}
							The idea is that for $f$ continuous, the preimage of an interval $I$ is a countable union of intervals $I_1,I_2,\hdots$. Since each interval $I_n$ is measurable (by question 2(a)), and a countable union of measurable sets is measurable (by question 1(b,iii)), this is enough to show that $f^{-1}(I)$ is measurable.
							
							So we just need to prove that the preimage of an interval under a continuous function is a countable union of intervals. This follows from some facts about topology, particularly something called the Heine-Borel theorem. Probably best to just take this as given! A sketch proof follows.\bigskip
							
							The idea is that if $x\in f^{-1}(I)$, where $I$ is an open interval $(a,b)$, then for any point $y$ near $x$, $f(y)$ must be near $f(x)$ (since $f$ is continuous); and any point sufficiently close to $f(x)$ will still be contained in $I$, because $I$ is an open interval, so does not include its endpoints. So, for instance, $f(x)$ could be very close to the edge of $I$, say $f(x)=b-\epsilon$ for some very small value of $\epsilon$, but still any point close enough to $x$ will map to a point within $\epsilon$ of $f(x)$, hence smaller than $b$; similarly if $x$ is close to $a$, $x=a+\epsilon$, any point $y$ close enough to $x$ will have $f(y)$ within $\epsilon$ of $x$ and hence bigger than $a$.
							
							So if $x\in f^{-1}(I)$, then there is some small interval around $x$ such that every point in that interval is also in $f^{-1}(I)$. Now, since the rational numbers are dense in $\mathbb{R}$ (meaning every real number can be approximated as close as we like by rational numbers), we can just take the rational numbers in $f^{-1}(I)$, and take a small interval around each of these rational numbers which is entirely contained within $f^{-1}(I)$. Then this gives us countably many intervals (since there are only countably many rationals), and the union of these intervals must include all of $f^{-1}(I)$, since any point in $f^{-1}(I)$ is either rational, or else comes as close as we like to a rational in $f^{-1}(I)$, so is in one of these intervals.
							
							So for any open interval $I$, $f^{-1}(I)$ can be written as a countable union of open intervals (one around each rational point in $f^{-1}(I)$). Then for a closed interval $J$, the complement $\mathbb{R}\take J$ is a union of two open intervals, so by taking preimages we can show that the complement of $f^{-1}(J)$ is a union of countably many open intervals, hence $f^{-1}(J)$ is the complement of a measurable set, hence is measurable itself by question 1(b,ii).
							
							This deals with open and closed intervals. For half-open intervals ($(a,b]$ and $[a,b)$), the complement is the union of an open interval and a closed interval, so again taking the preimage of the complement works.
						}
			
			\item If $f$ and $g$ are measurable, then so is $f(g(x))$.
			
						{\color{blue}
							Using part (a), this is easy. Take an interval $I$; the preimage under $f(g(x))$ is $g^{-1}(f^{-1}(I))$ (so first take the preimage under $f$, then take the preimage of that under $g$). Since $f$ is measurable, $f^{-1}(I)$ is measurable; since $g$ is measurable and using part (a), the preimage of $f^{-1}(I)$ is measurable. So the preimage of an interval $I$ under $f(g(x))$ is measurable, so $f(g(x))$ is a measurable function.
						}
					
			\item If $f$ and $g$ are measurable and $\alpha$ is a constant, then so are $f(x)+g(x)$, $f(x)g(x)$, and $\alpha f(x)$.
			
						{\color{blue}
							Each of these is easy by using parts (b) and (c). Technically we need to introduce Lebesgue measure for $\mathbb{R}^n$ to make this work, but its essentially the same as on $\mathbb{R}$, just replace intervals by rectangles/cuboids/etc.
							
							For $f+g$, we have the plus function $p:\mathbb{R}^2\to \mathbb{R}$ defined by $p(x,y)=x+y$; this is continuous, hence measurable, by (b). We also have a function $h:\mathbb{R}\to\mathbb{R}^2$, defined by $h(x)=(f(x),g(x))$; this is measurable because $f$ and $g$ are measurable. Then $f(x)+g(x)=p(h(x))$, which is measurable by part (c).
							
							Similarly for $f(x)g(x)$, but instead of $p$, we use the multiplication function $m$, defined by $m(x,y)=xy$. For $\alpha f(x)$, we just use the ``multiply by $\alpha$'' function $\mathbb{R}\to\mathbb{R}$.
							
							The point with all of these is that once you know continuous functions are measurable (b) and that doing two measurable functions in a row is measurable, then we can build all sorts of functions by combining continuous functions in various ways.
						}
					
			\item If $S$ is a Lebesgue-measurable subset of $\mathbb{R}$, then the indicator function $i_S(x)$ (1 when $x\in S$, 0 otherwise) is measurable.
			
						{\color{blue}
							Let $I$ be an interval; we need to show that $i_S^{-1}(x)$ is measurable. Since $i_S(x)$ only takes two values (0 and 1), we can split into four cases: when $I$ contains neither 0 nor 1, when it contains 0 but not 1, when it contains 1 but not 0, and when it contains both. In these four cases, the preimage of $I$ is, respectively, $\varnothing$, $\mathbb{R}\take S$, $S$, and $\mathbb{R}$. All of these are measurable, so whatever the interval we started with, the preimage is measurable.
						}
					
			\item If $f$ is measurable, then $f_+$ and $f_-$ are measurable, where
				\[f_+(x) = \frac{f(x)+|f(x)|}{2},\quad f_-(x)=\frac{f(x)-|f(x)|}{2}.\]
				
						{\color{blue}
							Adding, absolute value function, and dividing by 2 are all continuous; apply parts (b) and (c).
						}
					
		\end{enumerate}
		Similar results can be shown for complex-valued functions. In particular, if $f:I\to \mathbb{C}$ is a complex-valued, measurable function on an interval $I$, then $f(x)\overline{f(x}$ is measurable, since it is the product of $f$ with $g(f)$, where $g$ is the continuous function complex-conjugation. This means the integral in the definition of an $L_2$ function is well-defined.
	\item In this question we prove that $L_2(I)$ is an inner product space (for $I$ an interval in $\mathbb{R}$).
		\begin{enumerate}
			\item First we show that $L_2(I)$ is a $\mathbb{C}$-vector space. The results of the previous question, on measurable functions, will be useful here.
				\begin{enumerate}
					\item Show that if $f\in L_2(I)$ (so $f$ is measurable and $f(x)\overline{f(x)}$ has finite integral), then $\alpha f\in L_2(I)$ for any constant $\alpha\in \mathbb{C}$.
					
						{\color{blue}
							By question 4(d), $\alpha f$ is measurable, so we just need to check that
							\[\int_I(\alpha f(x))\overline{(\alpha f(x))}\diff x\]
							is finite. But $\alpha$ is a constant, so this is
							\[\alpha\overline{\alpha}\int_I f(x)\overline{f(x)}\diff x,\]
							which is finite because $f$ is $L_2$ (square-integrable).
						}
					
					\item Show that for any complex numbers $z$ and $w$, $|z+w|^2\leq 2|z|^2 + 2|w|^2$.
					
						{\color{blue}
							Let $z=a+bi$ and $w=c+di$. Then $z+w=(a+c)+(b+d)i$ and
							\begin{align*}
								(a-c)^2 + (b-d)^2&\geq 0\\
								a^2-2ac+c^2+b^2-2bd+d^2&\geq 0\\
								a^2+c^2+b^2+d^2&\geq 2ac+2bd\\
								2\left(a^2+c^2+b^2+d^2\right)&\geq a^2+2ac+c^2 + b^2+2bd +d^2\\
								2\left((a^2+b^2)+(c^2+d^2)\right) &\geq (a+c)^2+(b+d)^2\\
								2\left(|z|^2+|w|^2\right)&\geq |z+w|^2.
							\end{align*}
						}
					
					\item Hence show that, for $f$ and $g$ in $L_2(I)$, $f+g\in L_2(I)$.
						
						{\color{blue}
							By question 4(d), $f+g$ is measurable, so we just need to check that
							\[\int_I |f(x)+g(x)|^2\diff x\]
							is finite.
							
							For any $x\in I$, we can take $z=f(x)$, $w=g(x)$ in part (ii) to see that $|f(x)+g(x)|^2\leq 2|f(x)|^2+2|g(x)|^2$ (and both of these are non-negative, since they are moduli of complex numbers). Therefore the area under the graph of $|f(x)+g(x)|^2$ must be less than or equal to the area under the graph of $2|f(x)|^2+2|g(x)|^2$:
							\[\int_I|f(x)+g(x)|^2\diff x\leq 2\int_I |f(x)|^2\diff x + 2\int_I |g(x)|^2\diff x.\]
							The integrals on the right-hand side are both finite, because $f$ and $g$ are $L_2$, so the integral on the left-hand side is finite too.
						}
					
					\item Check the axioms of a vector space to complete the proof that $L_2(I)$ is a $\mathbb{C}$-vector space.
					
						{\color{blue}
							Routine.
						}
					
				\end{enumerate}
			\item Next we show that the inner product on $L_2(I)$ is well-defined. It follows from our results on measurable functions (assuming the definition of measurability for complex-valued functions) that $f(x)\overline{g(x)}$ is measurable, so we just need to show that it has a finite integral, then check the axioms of an inner product.
				\begin{enumerate}
					\item Show that for any complex numbers $z$ and $w$
						\[2|z||w|\leq |z|^2+|w|^2.\]
						Hint: consider $\left(|z|-|w|\right)^2$.
						
						{\color{blue}
							We have
							\begin{align*}
								\left(|z|-|w|\right)^2 &\geq 0\\
								|z|^2-2|z||w|+|w|^2&\geq 0\\
								|z|^2+|w|^2&\geq 2|z||w|.
							\end{align*}
						}
					
					\item Hence show that for $f,g\in L_2(I)$,
						\[\int_I f(x)\overline{g(x)}\diff x < \infty\]
						and therefore the inner product is well-defined.
						
						{\color{blue}
							Taking $f(x)=z$ and $g(x)=w$ in part (i), then integrating, we see that
							\[\int_I|f(x)||g(x)|\diff x \leq \frac{1}{2}\int_I |f(x)|^2\diff x + \frac{1}{2}\int_I |g(x)|^2\diff x.\]
							Both of the right-hand integrals are finite, since $f$ and $g$ are $L_2$, so the left-hand integral is finite. The left-hand integral is equal to
							\[\int_I |f(x)\overline{g(x)}|\diff x,\]
							by basic properties of the modulus function, so we just need to show that
							\[\left|\int_I f(x)\overline{g(x)}\diff x \right|\leq \int_I |f(x)\overline{g(x)}|\diff x.\]
							But this is true because $|f(x)\overline{g(x)}|\geq 0$, so this always contributes positively to the integral, whereas $f(x)\overline{g(x)}$ can be positive or negative (in both real and imaginary parts), so can give positive and negative areas that cancel out. For a more detailed proof, write out the definition of the integrals as limits of sums, and use the fact that $|z+w|\leq |z|+|w|$ in the sums (after checking that an infinite version of this inequality holds!).
						}
					
					\item Check the axioms of a complex inner product (linearity in the first argument, conjugate-symmetry, and positive-definiteness) to complete the proof that $L_2(I)$ is an inner product space.
					
						{\color{blue}
							Routine.
						}
					
				\end{enumerate}
		\end{enumerate}
	\item In this question we show that not all subsets of $\mathbb{R}$ are Lebesgue measurable. Therefore it really is necessary to introduce the machinery of $\sigma$-algebras to work with measures.
		\begin{enumerate}
			\item First we must prove a property of the Lebesgue measure called \textbf{translation-invariance}. This says that if $E\subseteq \mathbb{R}$ is a measurable set and $x\in\mathbb{R}$, then $x+E=\{x+e : e\in E\}$ is measurable and $\lambda(E)=\lambda(x+E)$---so adding a fixed amount onto every element of a set doesn't change its size.
				\begin{enumerate}
					\item First show that if $A$ is any subset of $\mathbb{R}$ and $(a_1,b_1)$, $(a_2,b_2)$, $\hdots$ is a (countable) sequence of intervals such that
						\[A\subseteq \bigcup_{n=1}^\infty (a_n,b_n),\]
						then
						\[x+A\subseteq \bigcup_{n=1}^\infty (a_n+x,b_n+x).\]
						So we can turn a cover of $A$ by open intervals into a cover of $x+A$, simply by translating the open intervals by $x$.
					\item Hence show that for any $A$ we have $\lambda^+(x+A)=\lambda^+(A)$.
					\item Hence show that if $E$ is Lebesgue measurable, so is $x+E$. Hint: show that $A\cap(x+E)=x+([(-x)+A]\cap E)$ and $\mathbb{R}\take(x+E)=x+(\mathbb{R}\take e)$.
					\item Hence conclude that $\lambda$ is translation-invariant.
				\end{enumerate}
			\item Now we construct a set, called a \textbf{Vitali set}, which we will show is not measurable. We start with the rationals $\mathbb{Q}$, and consider translations $x+\mathbb{Q}$.
				\begin{enumerate}
					\item Show that for any two real numbers $x$ and $y$, if $x-y\in \mathbb{Q}$, then $x+\mathbb{Q}=y+\mathbb{Q}$, and if $x-y\in\mathbb{R}\take\mathbb{Q}$, then $(x+\mathbb{Q})\cap(y+\mathbb{Q})=\varnothing$.
					\item Show that for any real $x$, $(x+\mathbb{Q})\cap [0,1]\neq \varnothing$ (\textit{i.e.}, there is always an element of $x+\mathbb{Q}$ between 0 and 1).
					\item Choose exactly one element from each $(x+\mathbb{Q})\cap[0,1]$ for $x\in\mathbb{R}$, and let $V$ be the set containing your choices (and nothing else). So $V$ is a subset of $[0,1]$ and for any two elements $x$ and $y$ of $V$, $x+\mathbb{Q}\neq y+\mathbb{Q}$. Note that the existence of $V$ depends on a mildly controversial axiom of set theory called the Axiom of Choice. Show that for any $q\in\mathbb{Q}$, $q+V$ contains exactly one rational number.
				\end{enumerate}
			\item Let $q_1,q_2,\hdots$ be an enumeration of the rational numbers in $[-1,1]$ (so every rational number in $[-1,1]$ appears exactly once in this list; this is possible because the rationals are countable). Let $V_i=V+q_i$ for each $i$. Show that $V_i\cap V_j=\varnothing$ for $i\neq j$.
			\item Now we show that
				\[[0,1]\subseteq \bigcup_{i=1}^\infty V_i\subseteq [-1,2].\]
				\begin{enumerate}
					\item For $r\in [0,1]$, there is a unique element $v$ of $r+\mathbb{Q}$ in $V$ (by definition of $V$). Show that $v-r\in\mathbb{Q}\cap [-1,1]$.
					\item Hence there is some $i$ such that $q_i=v-r$. Conclude that $r\in V_i$.
					\item Hence conclude that
						\[[0,1]\subseteq \bigcup_{i=1}^\infty V_i.\]
					\item Use the fact that $V\subseteq [0,1]$ (by definition) to show that
						\[\bigcup_{i=1}^\infty V_i\subseteq [-1,2].\]
				\end{enumerate}
			\item Now assume for a contradiction that $V$ is Lebesgue measurable. By translation-invariance, so is each $V_i$, and $\lambda(V_i)=\lambda(V)$. Explain why it follows that
				\[\lambda([0,1])\leq \sum_{i=1}^\infty \lambda(V)\leq \lambda([-1,2]).\]
			\item Deduce a contradiction, and hence conclude that $V$ is not in fact Lebesgue measurable.
		\end{enumerate}
\end{enumerate}














\end{document}