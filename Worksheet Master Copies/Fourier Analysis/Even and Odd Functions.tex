\documentclass{article}

\usepackage[left=2cm,right=2cm, top=2cm, bottom = 2cm]{geometry}
\usepackage{amsfonts}

\usepackage{amsmath}
\usepackage{xcolor}

\usepackage{tikz}
\usepackage{subfigure}



\pagestyle{empty}

\setlength{\tabcolsep}{15pt}


\newcommand{\deriv}[3][]{\frac{\mathrm{d}^{#1}#2}{\mathrm{d}#3^{#1}}}
\newcommand{\diff}{\;\mathrm{d}}

\newcommand{\norm}[1]{\left|\kern-1pt\left|#1\right|\kern-1pt\right|}




\begin{document}

\title{Even and Odd Functions}
\date{}

\maketitle
\thispagestyle{empty}

\Large

\textbf{\underline{Objective: To understand the decomposition of a function into even}}

\textbf{\underline{and odd parts.}}






\vspace{5mm}







\textbf{Theory: Even and Odd Functions:}\bigskip


Recall that a function $f(x)$ is \textbf{even} if $f(-x)=f(x)$ for any $x$, and is \textbf{odd} if $f(x)=-f(-x)$ for any $x$. The standard examples are that $x^n$ is odd for $n$ odd and even for $n$ even (hence the use of the same adjectives!), and the normal and hyperbolic sine and tangent functions are odd, while the normal and hyperbolic cosine functions are even.

Graphically, $f$ is even if and only if the graph of $y=f(x)$ is symmetrical under reflection in the $y$-axis; this is because reflection in the $y$-axis corresponds to switching $x$ and $-x$, but $f(x)=f(-x)$ for an even function, so the function is unchanged by the reflection.

For odd functions, on the other hand, if you reflect the part of the graph for positive $x$ in the $y$-axis and also the $x$-axis, you get the part of the graph for negative $x$. These two reflections have the same effect as a rotation by $180^\circ$ about the origin, so a function is odd if and only if it has rotational symmetry of $180^\circ$ about the origin.

Most functions are neither even nor odd. For instance, if $f(x)=x^2+x$, then $f(-1)=0$, whereas $f(1)=2$, so $f$ is neither even nor odd. However, we see that $f(x)$ can be written as an even function, $x^2$, plus an odd function, $x$. In fact, this is always true! For any function $f$, define the \textbf{even part} and \textbf{odd part} of $f$ by:
\begin{align*}
	f_\mathrm{even}(x)&=\frac{f(x)+f(-x)}{2}\\
	f_\mathrm{odd}(x)&=\frac{f(x)-f(-x)}{2}.
\end{align*}

Compare with the definitions of $\cosh(x)$ and $\sinh(x)$; these are precisely the even and odd parts respectively of the exponential function.



\clearpage






	
	
	






\textbf{Practice:}\bigskip



\begin{enumerate}
	\item Show that for any function $f$:
		\begin{enumerate}
			\item $f_\mathrm{even}$ is indeed even.
			\item $f_\mathrm{odd}$ is indeed odd.
			\item $f(x)=f_\mathrm{even}(x)+f_\mathrm{odd}(x)$.
		\end{enumerate}
	\item Let $f(x)=7x^4-9x^3+3x^2+12x+8$. Find the even and odd parts of $f$.
	\item Find the even and odd parts of $\cos\left(2\pi t + \frac{\pi}{3}\right)$.
	\item Let $f$ be an integrable function, and $a$ a positive constant. Show that:
		\begin{enumerate}
			\item If $f$ is even,
				\[\int_{-a}^a f(x)\diff x=2\int_0^a f(x)\diff x.\]
			\item If $f$ is odd,
				\[\int_{-a}^a f(x)\diff x=0.\]
			\item Hence, whatever $f$ is,
				\[\int_{-a}^a f(x)\diff x = 2\int_0^a f_\mathrm{even}(x)\diff x.\]
		\end{enumerate}
	\item Let $E_1$ \& $E_2$ be even functions and $O_1$ \& $O_2$ odd functions, with $E_2(x)$ and $O_2(x)$ not constantly zero. Show that:
		\begin{enumerate}
			\item $O_1(x)=0$.
			\item $E_1(x)E_2(x)$ and $O_1(x)O_2(x)$ are even.
			\item $E_1(x)O_1(x)$ is odd.
			\item $E_1(x)+E_2(x)$ is even.
			\item $O_1(x)+O_2(x)$ is odd.
			\item $E_2(x)+O_2(x)$ is neither even nor odd.
		\end{enumerate}
	\item Let $f(x)$ be a function which is both even and odd. Show that $f(x)=0$ for all $x$.
\end{enumerate}


















\clearpage




{\bf Key Points to Remember:}

\vspace{5mm}

\begin{enumerate}
	\item An \textbf{even function} $f$ is one such that $f(x)=f(-x)$.
	\item An \textbf{odd function} $f$ is one such that $f(x)=-f(-x)$.
	\item Any function $f$ can be decomposed into an \textbf{even part} and an \textbf{odd part}: $f(x)=f_\mathrm{even}(x)+f_\mathrm{odd}(x)$, where
		\begin{align*}
			f_\mathrm{even}(x)&=\frac{f(x)+f(-x)}{2}\\
			f_\mathrm{odd}(x)&=\frac{f(x)-f(-x)}{2}.
		\end{align*}
	\item Functions are usually neither even nor odd. The only function which is both even and odd is the zero function.
	\item For any integrable even function $E(x)$ and integrable odd function $O(x)$, and any positive constant $a$:
		\begin{align*}
			\int_{-a}^a E(x)\diff x &= 2\int_0^a E(x)\diff x\\
			\int_{-a}^a O(x)\diff x &= 0.
		\end{align*}
\end{enumerate}









\end{document}