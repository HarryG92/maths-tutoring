
\documentclass{article}

\usepackage[left=2cm,right=2cm, top=2cm, bottom = 2cm]{geometry}
\usepackage{amsfonts}

\usepackage{amsmath}
\usepackage{xcolor}

\usepackage{tikz}
\usepackage{subfigure}



\pagestyle{empty}

\setlength{\tabcolsep}{15pt}


\newcommand{\deriv}[3][]{\frac{\mathrm{d}^{#1}#2}{\mathrm{d}#3^{#1}}}
\newcommand{\diff}{\;\mathrm{d}}

\newcommand{\norm}[1]{\left|\kern-1pt\left|#1\right|\kern-1pt\right|}
\newcommand{\bra}[1]{\left\langle #1 \,\right|}
\newcommand{\ket}[1]{\left|\, #1\right\rangle}
\newcommand{\braket}[2]{\left\langle #1 \mid #2 \right\rangle}




\begin{document}

\title{Fourier Series}
\date{}

\maketitle
\thispagestyle{empty}

\Large

\textbf{\underline{Objective: To understand how an even periodic function can be}}

\textbf{\underline{approximated by a series of cosines, and any periodic function can}}

\textbf{\underline{be approximated by a series of sines and cosines.}}






\vspace{5mm}



\textbf{Recap/Warm-up: Fourier Sine Series and Orthonormality of Cosines:}\bigskip



Recall that the Fourier sine series of a function $f(x)$ on the interval $[a,a+L]$ is
\[f_\mathrm{sin}(x)=\sum_{n=1}^\infty b_n\sin\left(\frac{2\pi nx}{L}\right),\]
where
\[b_n=\frac{2}{L}\int_a^{a+L} f(x)\sin\left(\frac{2\pi nx}{L}\right)\diff x.\]
Outside the interval $[a,a+L]$, $f_\mathrm{sin}$ approximates the periodic extension of $f$.\bigskip


\begin{enumerate}
	\item Let $E(x)$ be an even function defined on $\left[-\frac{L}{2},\frac{L}{2}\right]$. Show that for all $n$, $b_n=0$.
	\item Show that the functions
		\[\cos\left(\frac{2\pi nx}{L}\right)\]
		for $n\geq 1$, and also the constant function $\frac{1}{2}$, are orthonormal with respect to the inner product on $[a,a+L]$ defined by
		\[\braket{f}{g}=\frac{2}{L}\int_a^{a+L}\!\!\! f(x)g(x)\diff x.\]
	\item Show that for any $n$ and $m$ (whether equal or not):
		\[\left\langle \sin\left(\frac{2\pi nx}{L}\right)\,\Bigg|\,\cos\left(\frac{2\pi mx}{L}\right)\right\rangle=0\]
		with respect to the integral inner product on $[a,a+L]$. That is, that the cosine and sine functions of any harmonics of $L$ are orthogonal. Hint: use question 1!
	\item Show also that the constant function $\frac{1}{2}$ is orthogonal to the sines of the harmonics of $L$.
\end{enumerate}







\clearpage















\textbf{Theory: Fourier Cosine Series:}

\bigskip



We have seen in the warm-up that the Fourier sine series of an even function (or rather a function whose periodic extension is even) is always 0; essentially, since the sine functions of various harmonics are all odd, they cannot ever give a good approximation to an even function. We also saw that the cosine functions
\[\cos\left(\frac{2\pi nx}{L}\right)\]
and the constant function $\frac{1}{2}$ are orthonormal; so we can approximate a function by a linear combination of cosines:
\[f(x)\approx \left\langle f(x)\,\Bigg|\,\frac{1}{2}\right\rangle\frac{1}{2} + \sum_{n=1}^N \left\langle f(x)\,\Bigg|\,\cos\left(\frac{2\pi nx}{L}\right)\right\rangle \cos\left(\frac{2\pi nx}{L}\right).\]
As with the Fourier sine series, we take the limit as $N$ tends to infinity. So we write the \textbf{Fourier cosine series} of $f(x)$:
\[f_\mathrm{cos}=\frac{a_0}{2}+\sum_{n=1}^\infty a_n\cos\left(\frac{2\pi nx}{L}\right),\]
where
\[a_n = \frac{2}{L}\int_a^{a+L}f(x)\cos\left(\frac{2\pi nx}{L}\right)\diff x.\]

Just as a Fourier sine series is zero for an even function, a Fourier cosine series is zero for an odd function:\medskip

Show that if $O(x)$ is an odd function on the interval $\left[-\frac{L}{2},\frac{L}{2}\right]$, then $a_n=0$ for all $n$.



\clearpage












\textbf{Theory: Fourier Series for Arbitrary Functions:}\bigskip


We have seen how an even function can be approximated by a series of cosines, and an odd function by a series of sines. It is now clear how to extend to a general (square-integrable) function: we know that any function $f(x)$ can be decomposed as $f(x)=f_\mathrm{even}(x)+f_\mathrm{odd}(x)$, where
\begin{align*}
	f_\mathrm{even}(x)&=\frac{f(x)+f(-x)}{2}\\
	f_\mathrm{odd}(x)&=\frac{f(x)-f(-x)}{2}.
\end{align*}
Then, when finding our Fourier coefficients,
\begin{align*}
	a_n=\frac{2}{L}\int_a^{a+L} f(x)\cos\left(\frac{2\pi nx}{L}\right)\diff x &= \frac{2}{L}\int_a^{a+L}f_\mathrm{even}(x)\cos\left(\frac{2\pi nx}{L}\right)\diff x\\
	b_n=\frac{2}{L}\int_a^{a+L}f(x)\sin\left(\frac{2\pi nx}{L}\right)\diff x &=\frac{2}{L}\int_a^{a+L}f_\mathrm{odd}(x)\sin\left(\frac{2\pi nx}{L}\right)\diff x.
\end{align*}
So the Fourier sine series of $f$ is the same as the Fourier sine series of $f_\mathrm{odd}$, and the Fourier cosine series of $f$ is the same as the Fourier cosine series of $f_\mathrm{even}$. Then we simply add these two series together to obtain the \textbf{Fourier series} of $f$ on the interval $[a,a+L]$:
\[f_\mathrm{Fourier}(x)=\frac{a_0}{2}+\sum_{n=1}^\infty \left[a_n\cos\left(\frac{2\pi nx}{L}\right)+b_n\sin\left(\frac{2\pi nx}{L}\right)\right].\]

As with the sine and cosine series, the general Fourier series is periodic of period dividing $L$; so outside the interval $[a,a+L]$, it approximates the periodic extension of $f$.

When computing the Fourier coefficients $a_n$ and $b_n$, we can either use $f$ in our inner product integral, or $f_\mathrm{even}$ and $f_\mathrm{odd}$ respectively, whichever is more convenient. For instance, if $f(x)=x^2+x$, the decomposition of $f$ into even and odd parts is very straightforward, so we would use the even part ($x^2$) to find the $a_n$ and the odd part ($x$) to find the $b_n$. However, for some functions, it may be easier simply to leave the original function in than find its even and odd parts and integrate with those; it's really a matter of what makes the integrals easiest to evaluate.


\clearpage










\textbf{Practice:}\bigskip

\[f_\mathrm{Fourier}(x)=\frac{a_0}{2}+\sum_{n=1}^\infty\left[ a_n\cos\left(\frac{2\pi nx}{L}\right)+ b_n\sin\left(\frac{2\pi nx}{L}\right)\right],\]
where
\begin{align*}
	a_n &=\frac{2}{L}\int_a^{a+L} f(x)\cos\left(\frac{2\pi nx}{L}\right)\diff x\\
	b_n &= \frac{2}{L}\int_a^{a+L} f(x)\sin\left(\frac{2\pi nx}{L}\right)\diff x.
\end{align*}

\begin{enumerate}
	\item Let $f(x)$ be the ``staircase'' function defined by
		\[f(x)=\begin{cases}
				0: & 0\leq x\leq 1\\
				1: & 1\leq x\leq 2\\
				2: & 2\leq x \leq 3\\
				3: & 3\leq x\leq 4.
			\end{cases}\]
		Sketch the graph of the periodic extension of $f$ and find its Fourier series.
	\item Let $f(t)$ be the decaying sinusoid function defined by
		\[f(x) = e^{-|t|}\cos(10\pi t)\]
		on the interval $[-1,1]$; the graph of $f$ is shown below. Find the Fourier series of $f$.
\end{enumerate}

\begin{center}
\begin{tikzpicture}
	\draw[->] (-5,0) -- (5,0);
	\node[right] at (5,0) {$t$};
	\draw[->] (0,-2) -- (0,2.5);
	\node[above] at (0,2.5) {$f(t)$};
	\node[left] at (0,2) {1};
	\node[below] at (5,0) {1};
	\node[below] at (-5,0) {-1};
	
	\draw[blue,thick,domain=0:5,samples=200] plot (\x, { 2*exp(-\x)*cos(6.28*\x r) });
	\draw[blue,thick,domain=-5:0,samples=200] plot (\x, { 2*exp(\x)*cos(6.28*\x r) });
\end{tikzpicture}
\end{center}


















\clearpage




{\bf Key Points to Remember:}

\vspace{5mm}

\begin{enumerate}
	\item The \textbf{Fourier series} of a function $f(x)$ on the interval $[a,a+L]$ is the trigonometric series
		\[f_\mathrm{Fourier}(x)=\frac{a_0}{2}+\sum_{n=1}^\infty \left[a_n\cos\left(\frac{2\pi nx}{L}\right) + b_n\sin\left(\frac{2\pi nx}{L}\right)\right],\]
		where
		\begin{align*}
			a_n &= \frac{2}{L}\int_a^{a+L} f(x)\cos\left(\frac{2\pi nx}{L}\right)\diff x\\
			b_n &= \frac{2}{L}\int_a^{a+L} f(x)\sin\left(\frac{2\pi nx}{L}\right)\diff x.
		\end{align*}
	\item The $N^\mathrm{th}$ partial sum of the Fourier series is the best approximation to $f$ by a linear combination of the orthonormal functions
		\[\frac{1}{2},\cos\left(\frac{2\pi x}{L}\right),\sin\left(\frac{2\pi x}{L}\right),\hdots, \cos\left(\frac{2\pi N x}{L}\right),\sin\left(\frac{2\pi Nx}{L}\right)\]
		(with respect to the integral inner product on $[a,a+L]$).
	\item The cosine coefficients $a_n$ depend only on the even part of the original function, while the sine coefficients $b_n$ depend only on the odd part.
\end{enumerate}









\end{document}