
\documentclass{article}

\usepackage[left=2cm,right=2cm, top=2cm, bottom = 2cm]{geometry}
\usepackage{amsfonts}

\usepackage{amsmath}
\usepackage{xcolor}

\usepackage{tikz}
\usepackage{subfigure}



\pagestyle{empty}

\setlength{\tabcolsep}{15pt}


\newcommand{\deriv}[3][]{\frac{\mathrm{d}^{#1}#2}{\mathrm{d}#3^{#1}}}
\newcommand{\diff}{\;\mathrm{d}}

\newcommand{\norm}[1]{\left|\kern-1pt\left|#1\right|\kern-1pt\right|}




\begin{document}

\title{Exponential Form of Fourier Series}
\date{}

\maketitle
\thispagestyle{empty}

\Large

\textbf{\underline{Objective: To be able to express a periodic function as a series of}}

\textbf{\underline{exponential functions.}}






\vspace{5mm}



\textbf{Recap/Warm-up: Fourier Series:}\bigskip



Recall that the Fourier series of a function $f(t)$ on the interval $[a,a+L]$ is
\[f_\mathrm{Fourier}(t)=\frac{a_0}{2}+\sum_{n=1}^\infty \left[a_n\cos\left(\frac{2\pi nt}{L}\right) + b_n\sin\left(\frac{2\pi nt}{L}\right)\right],\]
where
\begin{align*}
	a_n&=\frac{2}{L}\int_a^{a+L} f(t)\cos\left(\frac{2\pi nt}{L}\right)\diff t\\
	b_n&=\frac{2}{L}\int_a^{a+L} f(t)\sin\left(\frac{2\pi nt}{L}\right)\diff t.
\end{align*}
Outside the interval $[a,a+L]$, $f_\mathrm{Fourier}$ approximates the periodic extension of $f$.\bigskip


\begin{enumerate}
	\item Let $f(t)$ be the periodic extension of the function $\cos(t)$ on the interval $[0,\pi]$. Show that the Fourier series of $f$ is
		\[f_\mathrm{Fourier}(t) = \sum_{n=1}^\infty \frac{8n}{(4n^2-1)\pi}\sin(2nt).\]
	\item Use Euler's Equation to show that
		\[\sin(2nt) = \frac{e^{2njt} - e^{-2njt}}{2j}.\]
	\item Hence show that the Fourier series can be rewritten as
		\[f_\mathrm{Fourier}(t) = \sum_{n=-\infty}^\infty \frac{4n}{(4n^2-1)\pi j} e^{2njt}.\]
\end{enumerate}







\clearpage















\textbf{Theory: Exponential Fourier Series:}

\bigskip


We have seen how rearranging Euler's equation gives us the identities
\[\cos(t)=\frac{e^{jt}+e^{-jt}}{2},\qquad \sin(t)=\frac{e^{jt}-e^{-jt}}{2j}.\]

Using these, we can rewrite a Fourier series as a sum of exponentials; however, to do this, we must allow negative values of $n$, so our sum goes from $-\infty$ to $\infty$. Given a Fourier series
\[F(t)=\frac{a_0}{2}+\sum_{n=1}^\infty \left[a_n\cos\left(\frac{2\pi nt}{L}\right) + b_n\sin\left(\frac{2\pi nt}{L}\right)\right],\]
we substitute our above identities, and set $b_0=0$:
\begin{align*}
	F(t) &= \frac{a_0}{2} - \frac{b_0j}{2} + \sum_{n=1}^\infty \left[a_n\frac{e^{2\pi njt/L}+e^{-2\pi njt/L}}{2} + b_n\frac{e^{2\pi njt/L}-e^{-2\pi njt/L}}{2j}\right]\\
	&= \frac{a_0}{2} -\frac{b_0j}{2}+ \sum_{n=1}^\infty \left[\frac{a_n}{2}\left(e^{2\pi njt/L}+e^{-2\pi njt/L}\right) - \frac{b_nj}{2}\left(e^{2\pi njt/L}-e^{-2\pi njt/L}\right)\right]\\
	&= \frac{a_0-b_0j}{2} + \sum_{n=1}^\infty \frac{a_n-b_nj}{2}e^{2\pi njt/L} + \sum_{n=1}^\infty \frac{a_n+b_nj}{2}e^{-2\pi njt/L}
\end{align*}

Since we have negative powers of $e$ occuring, it makes sense to allow $n$ to take negative values; how do our Fourier coefficients behave with negative values of $n$? We see that, simply substituting $-n$ into the definition:
\begin{align*}
	a_{-n} &= \frac{2}{L}\int_a^{a+L} f(t)\cos\left(\frac{-2\pi nt}{L}\right)\diff t\\
	&=\frac{2}{L}\int_a^{a+L} f(t)\cos\left(\frac{2\pi nt}{L}\right)\diff t\\
	&=a_n,
\end{align*}
since cosine is an even function. For the sine coefficients, on the other hand:
\begin{align*}
	b_{-n} &= \frac{2}{L}\int_a^{a+L} f(t)\sin\left(\frac{-2\pi nt}{L}\right)\diff t\\
	&= -\frac{2}{L}\int_a^{a+L} f(t)\sin\left(\frac{2\pi nt}{L}\right)\diff t\\
	&=-b_n,
\end{align*}
since sine is an odd function. So we can extend our definition of the Fourier coefficients to negative values of $n$, with the simple relations $a_{-n}=a_n$, and $b_{-n}=-b_n$.

Therefore, in the second infinite series of our rearranged Fourier series above, we can write
\begin{align*}
	\sum_{n=1}^\infty \frac{a_n+b_nj}{2}e^{-2\pi njt/L}&=\sum_{n=1}^\infty \frac{a_{-n}-b_{-n}j}{2}e^{-2\pi njt/L}\\
	&= \sum_{n=-\infty}^1 \frac{a_n-b_nj}{2}e^{2\pi njt/L}.
\end{align*}
This makes the summands equal to those of the first infinite series, but the limits now go from $-\infty$ to $1$, instead of $1$ to $\infty$. The constant term also has the same form, so we can treat it as
\[\frac{a_n-b_nj}{2}e^{2\pi njt/L}\]
when $n=0$. Therefore every term we are adding has the same form, so we can combine them all into a single infinite sum:
\[F(t)=\sum_{n=-\infty}^\infty \frac{a_n-b_nj}{2}e^{2\pi njt/L}.\]\medskip


Now consider the coefficients in the above series. From the definitions, we have
\begin{align*}
	\frac{a_n-b_nj}{2} &= \frac{1}{2}\left[\frac{2}{L}\int_a^{a+L}\!\!f(t)\cos\left(\frac{2\pi nt}{L}\right)\diff t -\frac{2j}{L}\int_a^{a+L}\!\! f(t)\sin\left(\frac{2\pi nt}{L}\right)\diff t\right]\\
	&=\frac{1}{L}\int_a^{a+L}\!\!f(t)\left[\cos\left(\frac{2\pi nt}{L}\right) - j \sin\left(\frac{2\pi nt}{L}\right)\right]\diff t\\
	&=\frac{1}{L}\int_a^{a+L}\!\! f(t)e^{-2\pi njt/L}\diff t.
\end{align*}

So we define
\[c_n=\frac{1}{L}\int_a^{a+L}\!\!f(t)e^{-2\pi njt/L}\diff t,\]
and write the \textbf{exponential Fourier series} of $f$ over $[a,a+L]$ as
\[f_\mathrm{exp}(t)=\sum_{n=-\infty}^\infty c_n e^{2\pi njt/L}.\]




\clearpage












\textbf{Theory: The Coefficient Function:}\bigskip


Consier the exponential Fourier coefficients, $c_n$. We can view this as a function which takes an integer $n$ as input and returns a complex number $c_n$ as output. Since this depends on the original function $f$, let us denote this coefficient function $\hat{f}$: $\hat{f}(n)=c_n$.

This process of ``putting a hat'' on a function $f$ is therefore a type of \textbf{operator}. We have seen operators before, such as differentiation; just as a function takes a number and gives you a number, an operator takes a function and gives you a new function. In this case, we start with a function $f$ and obtain a function $\hat{f}$. There is a crucial difference between this ``hat operator'' and the differentiation operator, however. Differentiation takes a function that has real inputs and real outputs, and it gives a function out that has real inputs and real outputs. The hat operator, on the other hand, takes a function with real inputs and outputs, and gives out a function with integer inputs and complex outputs!

The original function $f$ often has time as an input, in applications. The integer inputs to $\hat{f}$ are the harmonics over $L$, so represent frequencies. We say that $f$ is a function on the \textbf{time domain}, whereas $\hat{f}$ is a function on the \textbf{frequency domain}. In general, any real number is possible as the frequency of a sinusoid (or a complex exponential!), but $\hat{f}$ takes only integer multiples of the fundamental frequency on $L$. We say $\hat{f}$ is a function on the \textbf{discrete frequency domain}.

We shall see a generalisation of Fourier series later, the Fourier transform, which allows any real frequency as input, so is a function on the entire frequency domain, not just the integer multiples of a fixed fundamental frequency. The Fourier transform is a complex-valued function on the \textit{continuous} frequency domain.

It is worth looking more closely at the definition of our hat operator. It takes a function $f(t)$ and returns the function $\hat{f}(n)$, where
\[\hat{f}(n)=\frac{1}{L}\int_a^{a+L}\!\! f(t)e^{-2\pi njt/L}\diff t.\]

So we introduce a new variable $n$, then integrate with respect to the old variable to leave an expression using only the new variable. We multiply $f(t)$ by a function $e^{-2\pi njt/L}$ which involves \textit{both $t$ and $n$}, then integrate in the time domain to leave a function in the frequency domain. So the function $e^{-2\pi njt/L}$ acts as a bridge between the time and frequency domains.

To recover the original function, via its exponential Fourier series, we multiply by a similar ``bridge'' function, $e^{2\pi njt/L}$, and sum over all value of $n$---we sum over the frequency domain. This leaves us with $f_\mathrm{exp}(t)$, a function in the time domain, which (for sufficiently nice $f$), will be equal to the original function $f$.


\clearpage










\textbf{Practice:}\bigskip

\[f_\mathrm{exp}(t)=\sum_{n=-\infty}^\infty c_ne^{2\pi njt/L},\]
where
\[c_n = \frac{1}{L}\int_a^{a+L}\!\! f(t)e^{-2\pi njt/L}\diff t.\]

\begin{enumerate}
	\item Let $f(t)$ be the square wave function obtained by periodic extension of the Heaviside function on $[-1,1]$. Find the exponential Fourier series of $f$.
	\item Let $T(t)$ be the triangle wave function obtained by periodic extension of $|t|$ on $[-1,1]$. Find the exponential Fourier series of $T$.
	\item Let $S(t)$ be the sawtooth function obtained by periodic extension of $t$ on $[-1,1]$. Find the exponential Fourier series of $S$.
\end{enumerate}



















\clearpage




{\bf Key Points to Remember:}

\vspace{5mm}

\begin{enumerate}
	\item The \textbf{exponential Fourier series} of a function $f(x)$ on the interval $[a,a+L]$ is the complex exponential series
		\[f_\mathrm{exp}(t)=\sum_{n=-\infty}^\infty c_ne^{2\pi njt/L},\]
		where
		\[c_n = \frac{1}{L}\int_a^{a+L}\!\! f(t)e^{-2\pi njt/L}\diff t.\]
	\item If we substitute for the complex exponentials with Euler's equation, and split the coefficients $c_n$ into real and imaginary parts $c_n=a_n-b_nj$, we recover the trigonometric Fourier series.
	\item There is an \textbf{operator} taking the real-valued \textbf{time-domain} function $f$ to the complex-valued \textbf{discrete frequency-domain} function $\hat{f}$, defined by
		\[\hat{f}(n)=c_n=\frac{1}{L}\int_a^{a+L}\!\!f(t)e^{-2\pi njt/L}\diff t.\]
\end{enumerate}









\end{document}