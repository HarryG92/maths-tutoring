\documentclass{article}

\usepackage[left=2cm,right=2cm, top=2cm, bottom = 2cm]{geometry}
\usepackage{amsfonts}
\usepackage{amsmath}
%%%\usepackage{array}

\usepackage{tikz}

\pagestyle{empty}

\setlength{\tabcolsep}{15pt}
%%%\renewcommand{\arraystretch}{2.5}

%%%\makeatletter
%%%\newcommand{\thickhline}{%
%%%    \noalign {\ifnum 0=`}\fi \hrule height 2pt
%%%    \futurelet \reserved@a \@xhline
%%%}
%%%\newcolumntype{!}{@{\hskip\tabcolsep\vrule width 2pt\hskip\tabcolsep}}
%%%\makeatother

\def\ihat{\hat{\i}}
\def\jhat{\hat{\j}}
\def\cosec{\mathrm{cosec}}

\begin{document}

\title{Reciprocal Trigonometric Functions.}
\date{}

\maketitle
\thispagestyle{empty}

\Large

\textbf{\underline{Objective: To know and apply reciprocal trig functions.}}




\vspace{5mm}


\textbf{The Reciprocal Trigonometric Functions:}

\vspace{5mm}

Historically, the main use of trigonometry for hundreds of years was navigation at sea. For this purpose, since calculators did not yet exist, tables of trig values were written up, so a navigator on a ship could measure the angle $\theta$ of a star above the horizon, say, then just look up $\cos(\theta)$, etc. in his trig tables. Since navigation formulae included expressions like $\frac{1}{2}(1-\cos(\theta))$, these were included in the tables---this one being called the ``haversine'' (half versed sine) of $\theta$. There are therefore about a dozen different trig functions, most of which are simple modifications of the standard 3 (sin, cos, and tan).

Most of these extra trig functions, like the haversine, are no longer in use, since they are easily calculated from the standard functions. However, as well as the standard 3 trig functions, there are an extra 3, the \textbf{reciprocal trig functions} still in use. They are the \textbf{secant}, \textbf{cosecant}, and \textbf{cotangent} functions, usually called sec, cosec, and cot, respectively. They are defined by
\begin{align*}
\sec(\theta)&=\frac{1}{\cos(\theta)}\\
\cosec(\theta)&=\frac{1}{\sin(\theta)}\\
\cot(\theta)&=\frac{1}{\tan(\theta)}=\frac{\cos(\theta)}{\sin(\theta)}
\end{align*}

Because these are defined in terms of sin, cos, and tan, they are easy to work with---any problem in sec, cosec, and cot can be turned into one with sin, cos, and tan. Indeed, even tan can be turned into sin and cos, since $\tan(\theta)=\frac{\sin(\theta)}{\cos(\theta)}$. So the only trig functions you ever really need are sine and cosine; however, it can be convenient to have tan, sec, cosec, and cot around to keep formulae a little simpler.

\clearpage





\textbf{Practice:}

\vspace{5mm}


\begin{align*}
\sec(\theta)&=\frac{1}{\cos(\theta)}\\
\cosec(\theta)&=\frac{1}{\sin(\theta)}\\
\cot(\theta)&=\frac{1}{\tan(\theta)}=\frac{\cos(\theta)}{\sin(\theta)}
\end{align*}


\vspace{5mm}

\begin{enumerate}
\item Starting with the pythagorean identity $\sin^2(\theta)+\cos^2(\theta)=1$:
	\begin{enumerate}
	\item Divide through by $\sin^2(\theta)$ to find an equation linking $\cot(\theta)$ and $\cosec(\theta)$.
	\item Divide through by $\cos^2(\theta)$ to find an equation linking $\tan(\theta)$ and $\sec(\theta)$.
	\end{enumerate}
\item Find $\sec\left(\frac{\pi}{4}\right)$.
\item Solve $\cosec(\theta)=\frac{2}{\sqrt{3}}$.
\item Sketch the graph of $\sec(\theta)$.
\item Sketch the graph of $\cosec(\theta)$.
\item Sketch the graph of $\cot(\theta)$.
\item Solve $5-6\sec(x) = \cos(x)$ for $0\leq x < 2\pi$.
\end{enumerate}










\clearpage


\textbf{Key Points to Remember:}

\begin{enumerate}
\item The \textbf{reciprocal trigonometric functions} are \textbf{secant} (sec), \textbf{cosecant} (cosec), and \textbf{cotangent} (cot), defined by:
	\begin{align*}
	\sec(\theta)&=\frac{1}{\cos(\theta)}\\
	\cosec(\theta)&=\frac{1}{\sin(\theta)}\\
	\cot(\theta)&=\frac{1}{\tan(\theta)}=\frac{\cos(\theta)}{\sin(\theta)}
	\end{align*}
\item The pythagorean identity $\sin^2(\theta)+\cos^2(\theta)=1$ gives rise to two new identities:
	\[1+\tan^2(\theta)=\sec^2(\theta),\]
	\[1+\cot^2(\theta)=\cosec^2(\theta).\]
\item Any problem involving sec, cosec, and cot can be solved by converting into a problem involving sin, cos, and tan (or just sin and cos).
\end{enumerate}




\end{document}