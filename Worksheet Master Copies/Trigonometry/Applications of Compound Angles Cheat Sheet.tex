\documentclass{article}

\usepackage[left=2cm,right=2cm, top=2cm, bottom = 2cm]{geometry}
\usepackage{amsfonts}
%%%\usepackage{array}
\usepackage{enumitem}
\usepackage{multicol}

\usepackage{tikz}

\pagestyle{empty}


\begin{document}

\title{Sums of Sinusoids ``Cheat Sheet''}
\date{}

\maketitle
\thispagestyle{empty}


\Large

A step-by-step guide to converting a sum of two sinusoids with the same frequency into a single sinusoid, with a running example of expressing
\[3\cos\left((2t+\frac{\pi}{3}\right))-\sin(2t)\]
in the form $R\sin(2t+\alpha)$:

\vspace{5mm}


\textbf{Step-by-step instructions:}

\begin{enumerate}
\item Expand out any phase shifts using compound angle formulae.
\item Expand the desired expression using the compound angle formula.
\item Compare the expressions obtained by the last two steps.
\item Deduce two equations relating $R$ and $\alpha$ to known quantities.
\item Square these two equations and add.
\item Use the pythagorean identity to simplify and get a value for $R^2$.
\item Take the positive square root (or negative, but positive makes step 9 slightly easier).
\item Going back to the equations from step 4, divide the sin equation by the cos equation to find $\tan(\alpha)$. Simplify if possible.
\item Using the equations from step 4 (and the fact that $R$ is positive), work out which quadrant $\alpha$ is in.
\item Use the previous two steps to find $\alpha$, taking arctan and possibly adding $\pi$.
\item Put the values for $R$ and $\alpha$ back into the desired expression, which is now equal to the original expression.
\item If desired, plug values into a calculator for a numerical answer.
\end{enumerate}

\clearpage



\textbf{Running example:}
\[3\cos\left((2t+\frac{\pi}{3}\right))-\sin(2t)\]

\begin{enumerate}
\item We have a phase-shifted term, $\cos\left(2t+\frac{\pi}{3}\right)$, which expands to
	\[\cos(2t)\cos\left(\frac{\pi}{3}\right)-\sin(2t)\sin\left(\frac{\pi}{3}\right)=\frac{1}{2}\cos(2t)-\frac{\sqrt{3}}{2}\sin(2t)\]
	So our expression becomes:
	\[\frac{3}{2}\cos(2t)-\frac{3\sqrt{3}}{2}\sin(2t)-\sin(2t)=\frac{3}{2}\cos(2t)-\frac{3\sqrt{3}-2}{2}\sin(2t).\]
\item Our desired expression is
	\[R\sin(2t+\alpha)=R\sin(\alpha)\cos(2t)+R\cos(\alpha)\sin(2t).\]
\item Compare
	\[\frac{3}{2}\cos(2t)-\frac{3\sqrt{3}-2}{2}\sin(2t)\quad\mbox{  and  }\quad R\sin(\alpha)\cos(2t)+R\cos(\alpha)\sin(2t).\]
\item Hence
	\[R\sin(\alpha)=\frac{3}{2}\qquad\mbox{ and }\qquad R\cos(\alpha)=\frac{2-3\sqrt{3}}{2}.\]
\item \[R^2\sin^2(\alpha)+R^2\cos^2(\alpha)=\left(\frac{3}{2}\right)^2+\left(\frac{2-3\sqrt{3}}{2}\right)^2=\frac{40-12\sqrt{3}}{4}.\]
\item \[R^2\sin^2(\alpha)+R^2\cos^2(\alpha)=R^2(\sin^2(\alpha)+\cos^2(\alpha))=R^2\]
	\[\mbox{ so}\qquad R^2=\frac{40-12\sqrt{3}}{4}.\]
\item \[R=\frac{\sqrt{40-12\sqrt{3}}}{2}.\]
\item \[\tan(\alpha)=\frac{\sin(\alpha)}{\cos(\alpha)}=\frac{R\sin(\alpha)}{R\cos(\alpha)}=\frac{3/2}{(2-3\sqrt{3})/2}=\frac{3}{2-3\sqrt{3}}.\]
\item Since $R\sin(\alpha)=\frac{3}{2}$, $\sin(\alpha)$ is positive. Since $R\cos(\alpha)=\frac{2-3\sqrt{3}}{2}$ and $3\sqrt{3}>2$, $\cos(\alpha)$ is negative. So $\alpha$ is in the top left 		quadrant.
\item Since $\alpha$ is in the top left quadrant, arctan will not give us the correct value, so we need to add $\pi$. So
	\[\alpha=\tan^{-1}\left(\frac{3}{2-3\sqrt{3}}\right)+\pi.\]
\item So we have
	\[3\cos\left((2t+\frac{\pi}{3}\right))-\sin(2t)=\frac{\sqrt{40-12\sqrt{3}}}{2}\sin\left(2t+\tan^{-1}\left(\frac{3}{2-3\sqrt{3}}\right)+\pi\right).\]
\item \[3\cos\left((2t+\frac{\pi}{3}\right))-\sin(2t)\approx 2.192\sin(2t+2.388).\]
\end{enumerate}


\end{document}