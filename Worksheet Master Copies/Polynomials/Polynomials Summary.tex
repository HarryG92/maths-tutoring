\documentclass{article}

\usepackage[left=2cm,right=2cm, top=2cm, bottom = 2cm]{geometry}
\usepackage{amsfonts}
\usepackage{array}

\setlength{\tabcolsep}{1.8cm}
\renewcommand{\arraystretch}{2.5}

\makeatletter
\newcommand{\thickhline}{%
    \noalign {\ifnum 0=`}\fi \hrule height 2pt
    \futurelet \reserved@a \@xhline
}
\newcolumntype{!}{@{\hskip\tabcolsep\vrule width 2pt\hskip\tabcolsep}}
\makeatother

\begin{document}

\title{Polynomials - Summary}
\date{}

\pagestyle{empty}

\maketitle

\Large

\section{Key Points - Fill in the Blanks:}

Fill in the blanks in the key points below with a word or phrase. The unblanked versions are on the next page. Bear in mind that in some cases there may be several ways to phrase things, so if what you put isn't exactly what I had, that doesn't mean it's necessarily wrong. Ask for any clarification! Note that the size of the blank does not indicate the size of the missing word or phrase!

\begin{enumerate}
\item A {\bf polynomial} is a function of a {\bf variable} (often $x$, but it could be any letter), which uses only addition, subtraction, ................... and ........................
\item If $f(x)$ is a polynomial and $b$ is a number, then $f(b)$ is the number found by .......................
\item If $p(x)$ is a polynomial, a {\bf root} of $p$ is a number $b$ such that .............
\item If $f(x)$ and $g(x)$ are polynomials, we say that $g$ is a {\bf factor} of $f$ if there is some third polynomial $h(x)$ such that .............
\item If $f(x)$ and $g(x)$ are polynomials, and $b$ and $r$ are numbers, such that $f(x)=g(x)(x-b)+r$, we call $r$ the .................. when $f$ is ............... by $(x-b)$.
\item The {\bf Polynomial Factor Theorem} says that if $f(x)$ is a polynomial and $b$ is a number, then the following two statements are equivalent: ................. and .................
\item When taking square roots of an expression like $x^2=9$, we must always remember ..........
\item When solving a quadratic equation by {\bf completing the square}, we first ignore the .......... term and focus on the ............ and ............. terms.
\end{enumerate}

\clearpage

\section{Key Points to Remember}

The statements from the previous page, with the blanks filled in.

\begin{enumerate}
\item A {\bf polynomial} is a function of a {\bf variable} (say $x$), which uses only addition, subtraction, \textit{multiplication} and \textit{raising the variable to positive, whole number powers}
\item If $f(x)$ is a polynomial and $b$ is a number, then $f(b)$ is the number found by \textit{substituting $b$ in place of $x$ in the expression for $f(x)$}.
\item If $p(x)$ is a polynomial, a {\bf root} of $p$ is a number $b$ such that $f(b)=0$.
\item If $f(x)$ and $g(x)$ are polynomials, we say that $g$ is a {\bf factor} of $f$ if there is some third polynomial $h(x)$ such that $f(x)=g(x)h(x)$.
\item If $f(x)$ and $g(x)$ are polynomials, and $b$ and $r$ are numbers, such that $f(x)=g(x)(x-b)+r$, we call $r$ the \textit{remainder} when $f$ is \textit{divided} by $(x-b)$.
\item The {\bf Polynomial Factor Theorem} says that if $f(x)$ is a polynomial and $b$ is a number, then the following two statements are equivalent: \textit{$b$ is a root of $f(x)$ (i.e., $f(b)=0$)} and \textit{$(x-b)$ is a factor of $f(x)$ (i.e., $f(x)=(x-b)g(x)$ for some $g(x)$)}.
\item When taking square roots of an expression like $x^2=9$, we must always remember \textit{to put plus or minus!}
\item When solving a quadratic equation by {\bf completing the square}, we first ignore the \textit{constant} term and focus on the \textit{linear} ($x$) and \textit{quadratic} ($x^2$) terms.
\end{enumerate}


\clearpage

\section{Revision Questions}

\begin{enumerate}
\item Solve the following linear equations:
	\begin{enumerate}
	\item $3x+4=1$
	\item $-9y+17=y+2$
	\item $\frac{t}{4}-1=2t+8$
	\end{enumerate}
\item Expand out the brackets in the following expressions:
	\begin{enumerate}
	\item $(x-y)(a+bx)$
	\item $(a-2b)^3$
	\item $(1-(2-s+t))(s^2-4(t+1))$
	\item $(x+3)(x-3)$
	\end{enumerate}
\item Factorise:
	\begin{enumerate}
	\item $x^2+4x+4$
	\item $6z^2-5z-4$
	\item $acx^2 + adxy -bcxy -bdy^2$
	\item $m^2-n^2$
	\end{enumerate}
\item Perform the following divisions with remainder:
	\begin{enumerate}
	\item Divide $x^2-19x+7$ by $x+3$.
	\item Divide $t^3-2t+4$ by $t-1$.
	\end{enumerate}
\item Solve the following quadratic equations:
	\begin{enumerate}
	\item $x^2+7x+1=0$
	\item $2t^2-5t=t^2-6$
	\item $z^2-a^2=0$ where $a$ is a constant
	\item $y^2-18y+81=0$
	\end{enumerate}
\item Let $f(s)=2s^3-4s^2-s+3$
	\begin{enumerate}
	\item Evaluate $f(1)$
	\item Hence solve $f(s)=0$.
	\end{enumerate}
\end{enumerate}





\clearpage

\section{Solutions}

It is possible I've made a mistake or two in these, so if your answer is different from mine and after checking you can't find a mistake in your work, ask me about it!

\begin{enumerate}
\item 
	\begin{enumerate}
	\item $x=-1$
	\item $y=1.5$
	\item $t=\frac{-36}{7}$.
	\end{enumerate}
\item
	\begin{enumerate}
	\item $ax + bx^2+ay - bxy$
	\item $a^3-6a^2b+12ab^2-8b^3$
	\item $s^3-4st-4s-s^2+8t+4-s^2t+4t^2$
	\item $x^2-9$ - this is called the {\bf difference of two squares} and is useful to remember: $(a+b)(a-b)=a^2-b^2$.
	\end{enumerate}
\item Some of these are really quite tricky...
	\begin{enumerate}
	\item $(x+2)^2$
	\item $(3z-4)(2z+1)$
	\item $(ax-by)(cx+dy)$
	\item $(m+n)(m-n)$
	\end{enumerate}
\item
	\begin{enumerate}
	\item $x-22$ remainder $-59$
	\item $t^2+t-1$ remainder 3
	\end{enumerate}
\item
	\begin{enumerate}
	\item $x = \frac{-7\pm 3\sqrt{5}}{2}$ (note: $3\sqrt{5}=\sqrt{45}$)
	\item $t=2$ or $3$
	\item $z=\pm a$
	\item $y=9$
	\end{enumerate}
\item
	\begin{enumerate}
	\item $f(1)=0$
	\item By the Polynomial Factor Theorem, $f(s)=(s-1)g(s)$ for some $g$. Dividing $f(s)$ by $s-1$ gives $g(s)=2s^2-2s-3$. Then solving $g(s)=0$ gives $s=1\pm\sqrt{7}$. So the 3 roots of $f$ are $1, 1+\sqrt{7}, 1-\sqrt{7}$.
	\end{enumerate}
\end{enumerate}

\end{document}