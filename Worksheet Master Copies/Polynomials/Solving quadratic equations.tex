\documentclass{article}

\usepackage[left=2cm,right=2cm, top=2cm, bottom = 2cm]{geometry}
\usepackage{amsfonts}
%%%\usepackage{array}

\usepackage{tikz}

\pagestyle{empty}

%%%\setlength{\tabcolsep}{1.8cm}
%%%\renewcommand{\arraystretch}{2.5}

%%%\makeatletter
%%%\newcommand{\thickhline}{%
%%%    \noalign {\ifnum 0=`}\fi \hrule height 2pt
%%%    \futurelet \reserved@a \@xhline
%%%}
%%%\newcolumntype{!}{@{\hskip\tabcolsep\vrule width 2pt\hskip\tabcolsep}}
%%%\makeatother

\begin{document}

\title{Solving Quadratic Equations}
\date{}

\maketitle

\Large

{\bf \underline{Objective: To be able to solve quadratic equations.}}

\vspace{5mm}


{\bf Recap of previous material:}

\vspace{5mm}

\begin{enumerate}
\item Factorise $x^2-6x+8$.
\item Hence write down the two roots of $x^2-6x+8$.
\item Let $f(x)=x^3-7x^2+14x-8$. Evaluate $f(1)$. Hence find $a$ such that $f(x)=(x-a)g(x)$ for some $g$.
\item With the notation as above, find $g$.
\item Hence factor $f$ completely into the form $f(x)=(x-a)(x-b)(x-c)$.
\item Hence write down all three roots of $f$.
\end{enumerate}


\clearpage

{\bf Warm-up:}

\vspace{5mm}

\begin{enumerate}
\item Solve $3x+4=11x-8$.
\item Solve $2x^2+4x-3=5x^2+4x-6$.
\item Expand $(x-1)^2$. Hence solve $x^2-2x+1=0$.
\item Expand $(x-3)^2$. Hence solve $x^2-6x+9=0$.
\item Solve $x^2-12x+36=0$.
\item Let $a$ be a constant. Solve $x^2-2ax+a^2=0$.
\end{enumerate}

\clearpage


{\bf Theory - Completing the Square:}

Solve $x^2+2x+1=0$. Now solve $x^2+2x-3=0$.

\vfill

Solve $3x^2-18x-48=0$.

\vfill
\clearpage


{\bf Practice:}

\vspace{5mm}

\begin{enumerate}
\item Solve $x^2+16x=36$.
\item Solve $4x^2-8x-7=0$.
\item Solve $x^2+14x+50=0$.
\item Let $b$ and $c$ be constants. Solve $x^2+bx+c=0$ in terms of $b$ and $c$.
\item Let $a$, $b$, and $c$ be constants. Solve $ax^2+bx+c=0$ in terms of $a$, $b$, and $c$.
\item Solve $2x^2+4x-3=0$ by completing the square. Now solve by the formula you derived in the last question. Compare your answers.
\end{enumerate}

\clearpage

{\bf Application - Projectile Motion:}

\vspace{5mm}

Suppose a particle is thrown from $2m$ above flat ground. Initially, its sideways speed is $5ms^{-1}$, and its vertical velocity is $5ms^{-1}$ upwards. Ignore air resistance, so the only force acting on the particle is gravity, causing it to accelerate downwards at a constant rate of $9.81ms^{-2}$.

\begin{center}
\begin{tikzpicture}
\draw[->] (0,0) -- (8,0);
\node[right] at (8,0) {$x$};
\draw[->] (0,0) -- (0,5);
\node[above] at (0,5) {$y$};
\node[below left] at (0,0) {0};
\draw[fill] (0,2) circle [radius=0.15];
\node[left] at (0,2) {$2$};
\draw[thick, ->] (0,2) -- (1,3);
\draw[dotted, thick, domain=0:6.6335] plot (\x, {2+\x-(9.81/50)*\x^2});
\draw[fill] (6.6335,0) circle [radius=0.1];
\node[below] at (6.6335,0) {$p$};
\end{tikzpicture}
\end{center}


Given that the equation of motion is $s=ut+\frac{1}{2}at^2$, where $s$ is the change in position, $u$ is initial velocity, $t$ is time, and $a$ is acceleration, write down an equation for the $x$-position of the particle at time $t$ and another equation for the $y$-position of the particle at time $t$.

\vspace{4cm}



Eliminate $t$ from these equations to find an equation linking $x$ and $y$.

\vspace{4cm}

Use this equation to find the position $p$ where the particle lands on the ground.


\clearpage


{\bf Key Points to Remember}

\vspace{5mm}

\begin{enumerate}
\item If $a$ is any constant, the unique solution to $x^2+2ax+a^2=0$ is $x=a$.
\item You can solve a quadratic equation (\textit{e.g.}, $ax^2+bx+c=0$ with $a\neq 0$) by completing the square:
	\begin{enumerate}
	\item Divide through by $a$ to get
	\[x^2+\frac{b}{a}x+\frac{c}{a}=0\]
	\item Compare with $\left(x+\frac{b}{a}\right)^2$:
	\[\left(x+\frac{b}{2a}\right)^2-\left(\frac{b}{2a}\right)^2+\frac{c}{a}=x^2+\frac{b}{a}x+\frac{c}{a}=0\]
	\item Rearrange:
	\[\left(x+\frac{b}{2a}\right)^2=\left(\frac{b}{2a}\right)^2-\frac{c}{a}=\frac{b^2-4ac}{4a^2}\]
	\item Take square roots:
	\[x+\frac{b}{2a}=\pm\frac{\sqrt{b^2-4ac}}{2a}\]
	\item Rearrange to get just $x$ on the left-hand side:
	\[x=\frac{-b\pm\sqrt{b^2-4ac}}{2a}\]
	\end{enumerate}
	{\bf Note: Completing the square is a \underline{process} best learned by repeated practice, not by trying to memorise the above instructions!}
\item The formula derived above can be used directly instead of the process of completing the square.
\end{enumerate}





\end{document}