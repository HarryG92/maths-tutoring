\documentclass{article}

\usepackage[left=2cm,right=2cm, top=2cm, bottom = 2cm]{geometry}
\usepackage{amsfonts}
%%%\usepackage{array}

\usepackage{amsmath}
\usepackage{xcolor}

\usepackage{tikz}
\usepackage{subfigure}

\pagestyle{empty}

\setlength{\tabcolsep}{15pt}
%%%\renewcommand{\arraystretch}{2.5}

%%%\makeatletter
%%%\newcommand{\thickhline}{%
%%%    \noalign {\ifnum 0=`}\fi \hrule height 2pt
%%%    \futurelet \reserved@a \@xhline
%%%}
%%%\newcolumntype{!}{@{\hskip\tabcolsep\vrule width 2pt\hskip\tabcolsep}}
%%%\makeatother

\newcommand{\deriv}[2]{\frac{\mathrm{d}#1}{\mathrm{d}#2}}




\begin{document}

\title{Functions}
\date{}

\maketitle
\thispagestyle{empty}

\Large





\textbf{Composition and Product of Functions:}

\vspace{5mm}

Let $f(x)=\sin(x)$, $g(x)=e^{2x}$, and $h(x)=(x+1)^2$. Find:

\begin{enumerate}
	\item $fg(x)$
	\item $gf(x)$
	\item $f(x)g(x)$
	\item $fh(x)$
	\item $hf(x)$
	\item $f(x)h(x)$
	\item $gh(x)$
	\item $hg(x)$
	\item $g(x)h(x)$
	\item $fgh(x)$
	\item $f(x)gh(x)$
	\item $fg(x)h(x)$
\end{enumerate}



\clearpage



\textbf{Domain and Range; Inverse Functions:}

\vspace{5mm}


\begin{enumerate}
	\item Define the domain and range of a function $f$.
	\item Define what it means for a function $g$ to be the inverse of a function $f$.
	\item Find the inverses of the following functions, specifying their domains:
		\begin{enumerate}
			\item $f(x)=2x-3$, $x\in\mathbb{R}$.
			\item $g(x)=\frac{x+1}{x-1}$, $x>1$.
			\item $h(t)=e^{4t+3}$, $t\leq 0$.
			\item $r(t)=\sin(2\pi t)$, $-\frac{1}{4}\leq t < \frac{1}{4}$.
			\item $s(x)=x^2-6x+5$, $x\leq 3$.
		\end{enumerate}
	\item The graph of a function $f(x)$ is shown below. Sketch the graph of $f^{-1}(x)$ on the same axes.
		\begin{center}
		\begin{tikzpicture}
			\draw[->] (-5,0) -- (5,0);
			\node[right] at (5,0) {$x$};
			\draw[->] (0,-5) -- (0,5);
			\node[above] at (0,5) {$y$};
			
			\draw[blue,thick,domain=-2:3,samples=100] plot (\x, { 0.02*\x^4+ 0.2*(\x-1)^3 + 2 });
		\end{tikzpicture}
		\end{center}
\end{enumerate}


\clearpage



\textbf{Transformations of Graphs:}

\vspace{5mm}


\begin{enumerate}
	\item \begin{enumerate}
		\item Sketch the graph of $y=\cos(-\theta)$.
		\item Hence sketch the graph of $y=\cos\left(\frac{\pi}{2}-\theta\right)$.
		\item Compare this with the graph of $\sin(\theta)$. Explain this result using SOHCAHTOA.
		\end{enumerate}
	\item \begin{enumerate}
		\item Sketch the graph of $y=x^2$.
		\item By completing the square, express $x^2-4x-12$ in the form $(x-a)^2+b$ for some constants $a$ and $b$.
		\item Hence sketch the graph of $y=x^2-4x-12$.
		\end{enumerate}
	\item Sketch the graph of $y=\tan(x)$. Hence sketch the graph of $y=\tan\left(x+\frac{\pi}{4}\right)-1$.
	\item Sketch the graphs of $\sin(x)$, $2\sin(x)$, and $\frac{1}{2}\sin(x)$ on the same axes, indicating which is which.
	\item Sketch the graphs of $\cos(x)$, $\cos(2x)$, and $\cos\left(\frac{x}{2}\right)$ on the same axes, indicating which is which.
\end{enumerate}




\end{document}