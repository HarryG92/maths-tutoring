\documentclass{article}

\usepackage[left=2cm,right=2cm, top=2cm, bottom = 2cm]{geometry}
\usepackage{amsfonts}
%%%\usepackage{array}

\usepackage{amsmath}
\usepackage{xcolor}

\usepackage{tikz}
\usepackage{subfigure}

\pagestyle{empty}

\setlength{\tabcolsep}{15pt}
\renewcommand{\arraystretch}{1.5}

%%%\makeatletter
%%%\newcommand{\thickhline}{%
%%%    \noalign {\ifnum 0=`}\fi \hrule height 2pt
%%%    \futurelet \reserved@a \@xhline
%%%}
%%%\newcolumntype{!}{@{\hskip\tabcolsep\vrule width 2pt\hskip\tabcolsep}}
%%%\makeatother

\newcommand{\deriv}[2]{\frac{\mathrm{d}#1}{\mathrm{d}#2}}




\begin{document}

\title{Arithmetic and Geometric Progressions}
\date{}

\maketitle
\thispagestyle{empty}

\Large




\textbf{Sequences and Series; Sigma Notation:}\bigskip


A \textbf{sequence} (or \textbf{progression}) is an infinite list of numbers, $a_0, a_1, a_2,a_3,\hdots$. Formally, it is a function from the natural numbers to the real numbers (or the complex numbers, for a sequence of complex numbers, or...). That is, if $(a_n)$ is a sequence, what we mean is that for each natural number $n$, there is a unique number $a_n$, the $n^\mathrm{th}$ term in the sequence. The indexing on a sequence can start from any number, though starting from 0 or 1 is most common.

When adding numbers together, Sigma notation is a convenient shorthand. If $(a_n)$ is a sequence, then we write
\[\sum_{n=k}^{k+m} = a_k+a_{k+1}+\hdots+a_{k+m}\]
for the sum of the $k^\mathrm{th}$ to $(k+m)^\mathrm{th}$ terms inclusive. For instance
\[\sum_{n=3}^7 \left(2n-1\right) = 5 + 7 + 9 + 11 + 13 = 45.\]

A \textbf{series} is an ``infinite sum,'' made by adding together \textit{every} term of a sequence:
\[\sum_{n=0}^\infty a_n = a_0+a_1+a_2+\hdots\]
What do we mean by adding together infinitely many numbers? The idea is that if there is some value $S$ such that we can get as close as we like to $S$ by adding enough (but finitely many!) terms of the series, then we can consider that $S$ to be the value of the series. Not every series has a value; for instance
\[\sum_{n=1}^\infty n = 1+2+3+\hdots\]
grows larger than any value, so does not settle down to any value $S$; we say this series \textbf{diverges}. If the series has a value, we say it \textbf{converges}.






\clearpage

\textbf{Theory: Arithmetic Sequences and Series:}\bigskip


An \textbf{arithmetic sequence} $(a_n)$ is a sequence such that $a_{n+1}-a_n$ has the same value for all $n$, called the \textbf{common difference}, $d$. We often denote the first term simply $a$, so the sequence is
\[a, a+d, a+2d, a+3d,\hdots\]
The $n^\mathrm{th}$ term of the sequence is therefore $a_n=a+(n-1)d$. What about summing terms in an arithmetic sequence? Let $S_n$ denote the sum of the first $n$ terms of an arithmetic sequence:
\[S_n = \sum_{k=1}^n a+(k-1)d.\]

By considering the sum written out forwards and backwards, show that
\[2S_n = (2a+(n-1)d)n.\]

\vfill

Hence conclude that
\[S_n=\frac{n}{2}(2a+(n-1)d).\]

\vfill

Hence write down a formula for the sum of the first $n$ natural numbers.




\clearpage




\textbf{Practice: Arithmetic Progressions:}\bigskip


\begin{enumerate}
	\item Write down the expression for the $n^\mathrm{th}$ term of an arithmetic sequence with first term $-7$ and common difference $3$. Find the $8^\mathrm{th}$ term.
	\item Find the sum of the integers from $100$ to $1000$ inclusive.
	\item Find the sum of the first 30 odd numbers.
	\item An arithmetic progression has first term $12$ and common difference $18.4$. What is the first term greater than 100? After how many terms does the sum first exceed 500?
	\item You loan a friend $\pounds 250$ interest-free. They repay in installments; the first payment is $\pounds 5$, and each subsequent payment is $\pounds 4$ more than the last one. After how many payments will you be fully paid back? Should the last payment follow the same pattern as the previous ones in order to pay you back exactly?
	\item Let $(a_n)$ be an arithmetic sequence, such that the series
		\[\sum_{n=1}^\infty a_n\]
		converges. What must the values of $a$ and $d$ be? What therefore is the value of the series?
\end{enumerate}



\clearpage



\textbf{Theory: Geometric Sequences and Series:}\bigskip


A \textbf{geometric} sequence is a sequence $(a_n)$ where $\frac{a_{n+1}}{a_n}$ has the same value $r$ for all $n$, called the \textbf{common ratio}. As with arithmetic sequences, we typically denote the first term $a$, so the sequence is
\[a, ar, ar^2, ar^3,\hdots\]
The $n^\mathrm{th}$ term is therefore $ar^{n-1}$. What about summing terms in a geometric sequence? Let $S_n$ denote the sum of the first $n$ terms of a geometric sequence:
\[S_n = \sum_{k=1}^n ar^{k-1}.\]
By considering $S_n-rS_n$, show that
\[S_n = a\frac{1-r^{n}}{1-r}.\]

\vfill

Consider a geometric sequence with first term $100$ and common ratio $0.1$. Find $S_n$. As $n$ gets larger, what value does $S_n$ approach? What therefore is the value of the infinite series
\[\sum_{n=1}^\infty 100\times0.1^{n-1}?\]

\vfill


Generalise the above to find when a geometric series is convergent, and the value it converges to.






\clearpage


\textbf{Practice: Geometric Sequences and Series:}\bigskip

\begin{enumerate}
	\item Write down the expression for the $n^\mathrm{th}$ term of a geometric sequence with first term $3$ and common ratio $5$. Find the $3^\mathrm{rd}$ term.
	\item Find the sum of the first 10 terms of the geometric progression with first term $273$ and common ratio $-\frac{1}{3}$. Find the sum to infinity of the same series.
	\item A savings account is set up with $\pounds5000$. It pays $1\%$ interest \textit{per annum} (believe it or not, there was a time this might have been a realistic question!). No money is withdrawn. After how many years will the interest payment first exceed $\pounds100$? After how many years will the account balance first exceed $\pounds10000$?
	\item There is a classic (and bad) maths joke that begins as follows: an infinite number of mathematicians walk into a bar; the first orders a pint, the second orders half a pint, the third orders a quarter of a pint, and so on. Find the punchline by determining how many pints the bartender should pour, assuming the mathematicians don't mind sharing glasses.
	\item
		\begin{enumerate}
			\item Consider a geometric sequence $a_n=ar^{n-1}$. Let $b_n=\ln(a_n)$. Show that $(b_n)$ is an arithmetic sequence. Find its first term and common difference.
			\item Consider an arithmetic sequence $a_n=a+(n-1)d$. Let $b_n=e^{a_n}$. Show that $(b_n)$ is a geometric sequence. Find its first term and common ratio.
		\end{enumerate}
	\item A small number $P_0$ of rabbits is introduced to an island with plentiful vegetation. Let the rabbit population after $n$ years be $P_n$. The birth rate is $b$ and the death rate is $d$, both in rabbits per capita per annum, meaning that for each rabbit in the population, on average $b$ new rabbits will be born in a year, and $d$ will die. So the population changes with time according to the equation
		\[P_{n+1}=P_n - dP_n + bP_n.\]
		\begin{enumerate}
			\item Show that the rabbit population is a geometric sequence. Write down the first term and common ratio.
			\item If $P_0=10$, $b=1.4$ and $d=0.6$, find the rabbit population after 5 years.
			\item After 5 years, a deadly disease breaks out on the island, increasing the death rate to 2.5 (the birth rate remains at 1.4). If this continues, how long until the rabbit population goes extinct?
		\end{enumerate}
\end{enumerate}


\clearpage


\textbf{Advanced Questions:}\bigskip

Let $M$ be an $m\times m$ matrix and $v$ a length-$m$ column vector. Define a geometric vector sequence by $v_n=M^{n-1}v$:
\[v, Mv, M^2v, M^3v, \hdots\]
Let $s_n$ be the vector found by adding together the first $n$ terms of this sequence:
\[s_n=\sum_{k=1}^n M^{k-1}v.\]
\begin{enumerate}
	\item By considering $s_n-Ms_n$, show that
		\[(I_m-M)s_n = (I_m-M^n)v,\]
		where $I_m$ is the $m\times m$ identity matrix.
	\item Hence find $s_n$ in the case that $I_m-M$ is invertible.
	\item Suppose that $M^n$ tends to $0_m$ (the $m\times m$ zero matrix) as $n\to \infty$. Show that $(I_m-M)s_\infty = v$.
	\item Let
		\[v=\left(\begin{array}{c}1\\3\end{array}\right), \quad M=\left(\begin{array}{cc} 0 & \frac{-1}{2}\\ \frac{-1}{3} & 0\end{array}\right).\]
		Prove by induction that
		\[M^n = \left(\begin{array}{cc}\frac{1+(-1)^n}{\sqrt{2^{n+2}3^n}} & \frac{-1+(-1)^n}{\sqrt{2^{n+3}3^{n-1}}} \\ \frac{-1+(-1)^n}{\sqrt{2^{n+1}3^{n+1}}} &  \frac{1+(-1)^n}{\sqrt{2^{n+2}3^n}} \end{array}\right).\]
		Hence show that the sum of the geometric series with first term $v$ and common ratio $M$ is
			\[\left(\begin{array}{c} \frac{-3}{5}\\ \frac{16}{5}\end{array}\right).\]
\end{enumerate}


\end{document}