\documentclass{article}

\usepackage[left=2cm,right=2cm, top=2cm, bottom = 2cm]{geometry}
\usepackage{amsfonts}
%%%\usepackage{array}

\usepackage{amsmath}
\usepackage{xcolor}

\usepackage{tikz}
\usepackage{subfigure}

\pagestyle{empty}

\setlength{\tabcolsep}{15pt}
%%%\renewcommand{\arraystretch}{2.5}

%%%\makeatletter
%%%\newcommand{\thickhline}{%
%%%    \noalign {\ifnum 0=`}\fi \hrule height 2pt
%%%    \futurelet \reserved@a \@xhline
%%%}
%%%\newcolumntype{!}{@{\hskip\tabcolsep\vrule width 2pt\hskip\tabcolsep}}
%%%\makeatother

\newcommand{\deriv}[2]{\frac{\mathrm{d}#1}{\mathrm{d}#2}}




\begin{document}

\title{Sequences and Series}
\date{}

\maketitle
\thispagestyle{empty}

\Large





\textbf{Binomial Series:}

\vspace{5mm}


\begin{enumerate}
	\item Write out the binomial expansion of $(a+b)^n$.
	\item Write out Pascal's triangle as far as the level which begins $1,6,\hdots$
	\item Expand $(x+1)^6$.
	\item Expand $2^n=(1+1)^n$ by the binomial theorem. Hence write down the sum of the $n^\mathrm{th}$ row of Pascal's triangle.
	\item \begin{enumerate}
			\item Expand $(1+x)^{-1/2}$ up to and including the term in $x^3$.
			\item If we substitute $x=10$, then $(1+x)^{-1/2}=\frac{1}{\sqrt{11}}$; do you expect the series expansion with $x=10$ to give a good approximation to $\frac{1}{\sqrt{11}}$? Why/why not? Plug in $x=10$ to your series expansion and compare with what your calculator gives you for $\frac{1}{\sqrt{11}}$.
			\item Show that
				\[\frac{3}{10}\left(1-\frac{1}{100}\right)^{-1/2}=\frac{1}{\sqrt{11}}.\]
			\item Substitute $x=-\frac{1}{100}$ into your series expansion for $(1+x)^{-1/2}$ and multiply by $\frac{3}{10}$ to estimate $\frac{1}{\sqrt{11}}$. Compare with the answer given by your calculator.
		\end{enumerate}
	\item Expand $(2+x)^{3/5}$ up to and including the term in $x^3$. For what values of $x$ is this expansion valid?
\end{enumerate}



\clearpage




\textbf{Arithmetic Series:}\bigskip


\begin{enumerate}
	\item Write down the expression for the $n^\mathrm{th}$ term of an arithmetic sequence with first term $-7$ and common difference $3$. Find the $8^\mathrm{th}$ term.
	\item Let $(a_k)$ be an arithmetic progression. Write down the formula for
		\[\sum_{k=1}^n a_k\]
		and prove it.
	\item Find the sum of the first 1000 integers.
	\item An arithmetic progression has first term $12$ and common difference $18.4$. What is the first term greater than 100? After how many terms does the sum first exceed 500?
\end{enumerate}



\clearpage



\textbf{Geometric Series:}\bigskip


\begin{enumerate}
	\item Write down the expression for the $n^\mathrm{th}$ term of a geometric sequence with first term $3$ and common ratio $5$. Find the $3^\mathrm{rd}$ term.
	\item Let $(a_k)$ be a geometric progression. Write down the formula for
		\[\sum_{k=1}^n a_k\]
		and prove it.
	\item Find the sum of the first 10 terms of the geometric progression with first term $273$ and common ratio $-\frac{1}{3}$. Find the sum to infinity of the same series.
	\item A savings account is set up with \pounds5000. It pays 4\% interest \textit{per annum} (believe it or not, there was a time this might have been a realistic question!). No money is withdrawn. After how many years will the interest payment first exceed \pounds100? After how many years will the account balance first exceed \pounds10000?
	\item There is a classic (and bad) maths joke that begins as follows: an infinite number of mathematicians walk into a bar; the first orders a pint, the second orders half a pint, the third orders a quarter of a pint, and so on. Find the punchline by determining how many pints the bartender should pour, assuming the mathematicians don't mind sharing glasses.
\end{enumerate}




\end{document}